\documentclass{article}
\usepackage[T1]{fontenc}
\usepackage{babel}[ngerman]
\usepackage{geometry}
\usepackage{hyperref}
\usepackage{mathtools}
\usepackage{array}
\usepackage{systeme}

\usepackage{multirow, multicol}

\setcounter{secnumdepth}{0}

\usepackage{hyphenat}
\hyphenation{Mathe-matik wieder-gewinnen}

\usepackage{amssymb, amsmath, amsthm}
\usepackage{tcolorbox}
\usepackage{nicefrac}

\usepackage{tcolorbox}
\tcbuselibrary{theorems,skins,breakable}

\makeatletter
\newcounter{elimination@steps}
\newcolumntype{R}[1]{>{\raggedleft\arraybackslash$}p{#1}<{$}}
\def\elimination@num@rights{}
\def\elimination@num@variables{}
\def\elimination@col@width{}
\newenvironment{elimination}[4][0]
{
    \setcounter{elimination@steps}{0}
    \def\elimination@num@rights{#1}
    \def\elimination@num@variables{#2}
    \def\elimination@col@width{#3}
    \renewcommand{\arraystretch}{#4}
    \start@align\@ne\st@rredtrue\m@ne
}
{
    \endalign
    \ignorespacesafterend
}
\newcommand{\eliminationstep}[2]
{
    \ifnum\value{elimination@steps}>0\longrightarrow\quad\fi
    \left(
        \ifnum\elimination@num@rights>0
            \begin{array}
            {@{}*{\elimination@num@variables}{R{\elimination@col@width}}
            |@{}*{\elimination@num@rights}{R{\elimination@col@width}}}
        \else
            \begin{array}
            {@{}*{\elimination@num@variables}{R{\elimination@col@width}}}
        \fi
            #1
        \end{array}
    \right)
    & 
    \begin{array}{l}
        #2
    \end{array}
    &%                                    moved second & here
    \addtocounter{elimination@steps}{1}
}
\makeatother

\definecolor{royalazure}{rgb}{0.0, 0.22, 0.66}

\tcolorboxenvironment{theorem}{
enhanced jigsaw,colframe=royalazure,interior hidden,
breakable,before skip=10pt,after skip=10pt }

\tcolorboxenvironment{definition}{
enhanced jigsaw,colframe=cyan,interior hidden,
breakable,before skip=10pt,after skip=10pt }

\theoremstyle{definition}
\newtheorem*{definition}{Definition}

\theoremstyle{definition}
\newtheorem*{theorem}{Satz}

\theoremstyle{remark}
\newtheorem*{example}{Beispiel}

\theoremstyle{remark}
\newtheorem*{corollary}{Korollar}

\newcommand{\norm}[1]{\left\lVert #1 \right\rVert}
\newcommand{\inner}[2]{\left\langle #1,#2 \right\rangle}

\title{Fragenkatalog Ausarbeitung\\
\large Lineare Algebra 2024}

\begin{document}
    \maketitle
    \tableofcontents
    \pagebreak
    \section{Frage 3}
\textit{Das Produkt von Matrizen: Definition und Motivation durch
lineare Abbildung.}

\begin{definition}[Matrix-Multiplikation]
    Sei $A\in M(m\times k), B\in M(k\times n)$.
    \[
        A\cdot B = (c_{ij})\in M(m\times n)  
    \]
    mit
    \[
        c_{ij} :=a_{i1}b_{1j} + a_{i2}b_{2j}+ \dots a_{ik}b_{kj}
    = \sum_{l=1}^{k}a_{il}b_{lj}
    \]
\end{definition}

\textbf{Motivation durch lineare Abbildung}. Seien $A\in M(m\times k)$,
$B\in~M(k\times~n)$ Matrizen und $f_A, f_B$ die entsprechenden 
linearen Abbildungen. Wir definieren $A\cdot B$ so, dass
\[
    f_{AB}=f_A\circ f_B
.\]
Sei $e_j$ der $j$-te Einheitsvektor:
\[
    e_j=\begin{pmatrix}
        0\\ \vdots\\ 1\\ \vdots\\ 0
    \end{pmatrix}
.\]
Setzt man $e_j$ in $f_{AB}$ ein, so erhält man
\begin{align*}
    f_A \circ f_B (e_j) &= f_A(f_B(e_j)) = f_A(Be_j)\\
    &=f_A=\begin{pmatrix}
        b_{1j}\\ \vdots\\ b_{kj}
    \end{pmatrix} = A \begin{pmatrix}
        b_{1j}\\ \vdots\\ b_{kj}
    \end{pmatrix} = \begin{pmatrix}
        a_{11}b_{1j} + \dots + a_{1k}b_{kj}\\
        \vdots\\
        a_{i1}b_{1j} + \dots + a_{ik}b_{kj}\\
        \vdots\\
        a_{m1}b_{1j} + \dots + a_{mk}b_{kj}
    \end{pmatrix}\\
    &= f_{AB}(e_j) = j\text{-te Spalte der Matrix }AB
.\end{align*}
    \pagebreak
    \section{Frage 4}
\textit{Wie berechnet man die Anzahl der Wege zwischen zwei Ecken in
einem gerichteten Graphen? Formulierung und beweisen Sie den entsprechenden
Satz.}

\begin{theorem}[Anzahl der Wege mit Länge $k$]
    Sei $G=(V,E)$ ein gerichteter Graph mit Adjazenten-Matrix $A$. Dann ist
    die Anzahl der Wege von $x_i$ nach $x_j$ ($x_i,x_j\in V$) der Länge $k$
    der Koeffizient der Matrix $A^k=A\cdot A \cdots$ an der Stelle $(i,j)$.
\end{theorem}

\begin{proof}
    Anzahl der Wege mit Länge $k$.

    Beweis über Induktion nach $k$.\\
    \underline{$k = 1$} Gilt aufgrund der Definition der Adjazenten-Matrix.\\
    \underline{$k \to k + 1$} Angenommen der Satz gilt für $k$. Wir betrachten
    einen Weg der Länge $k+1$ von $x_i$ nach $x_j$
    \[
        x_i \to \underbrace{x_e \to \cdots \to x_j}_{\text{Weg der Länge }k}  
    \]
    für ein $x_e$, also $a_{ie} = 1$. Sei $A^k=(b_{pq})\in M(n\times n)$.
    Nach der Induktionshypothese gibt es $b_{ej}$ Wege der Länge $k$ von 
    $x_e$ nach $x_j$, also $b_{ej}$ Wege der Länge $k+1$ von $x_i$ 
    über $x_e$ nach $x_j$. Wir betrachten alle möglichen Wege über $x_e$ für
    $e=1,\dots,n$.

    Es gibt $\sum_{e=1}^{n}a_{ie}b_{ej}$ Wege der Länge $k+1$ von $x_i$ nach
    $x_j$. $\sum_{e=1}^{n}a_{ie}b_{ej}$ ist aber der Koeffizient von
    $A\cdot A^k=A^{k+1}$ an der Stelle $(i,j)$. Das heißt die Anzahl der Wege
    von $x_i$ nach $x_j$ der Länge $k+1$ ist der Koeffizient der Matrix
    $A^{k+1}$ an der Stelle $(i,j)$.
\end{proof}
    \pagebreak
    \section{Frage 5}
\textit{Der Gauß-Algorithmus: Beschreiben Sie genau wie man eine
Matrix auf die Gauß-Normalform bringt. Wie schaut dann die Lösung aus?}

\begin{definition}[Elementare Zeilen (Spalten) Umformungen]
    Sei\\ $A\in M(m\times~n)$. Es gibt 3 Typen von elementaren
    Zeilen (Spalten) Umformungen.
    \begin{enumerate}
        \item[(I)] Vertauschen von zwei Zeilen (Spalten).
        \item[(II)] Multiplikation einer Zeile (Spalte) mit einer Zahl 
        $\lambda\neq 0$. 
        \item[(III)] Addieren eines beliebigen Vielfachen einer Zeile (Spalte)
        zu einer anderen Zeile (Spalte).
    \end{enumerate}
\end{definition}

Sei $Ax=b$ ein Lineares Gleichungssystem mit der Form
\[
    \begin{pmatrix}
        a_{11} & \cdots & a_{1n}\\
        \vdots & \ddots & \vdots\\
        a_{m1} & \cdots & a_{mn}
    \end{pmatrix} \begin{pmatrix}
        x_1\\ \vdots\\ x_n
    \end{pmatrix} = \begin{pmatrix}
        b_1\\ \vdots\\ b_n
    \end{pmatrix}
.\]
Zuerst muss die Matrix $A$ um den inhomogenen Vektor $b$ erweitert werden
\[
    (A,b):= \begin{pmatrix}
        a_{11} & \cdots & a_{1n} & \mid & b_1\\
        \vdots & \ddots & \vdots & \mid & \vdots\\
        a_{m1} & \cdots & a_{mn} & \mid & b_n
    \end{pmatrix}
.\]
Anschließend muss die Matrix durch elementare Umformungen in 
\textbf{Halbdiagonalen Form} gebracht werden, wobei die Spalte
des inhomogenen Vektors nicht vertauscht werden darf.

\underline{Halbdiagonalen-Form}: Sei $A\neq0 \implies \exists a_{ik} \neq 0$.
Dann durch Zeilen- und Spalten-Vertauschen $a_{ik}$ an die Stelle $(1,1)$ 
bringen. Es kann also angenommen werden, dass $a_{11} \neq 0$. 

Multipliziere die erste Zeile mit $\displaystyle\frac{1}{a_{11}}$. 
Dadurch lautet die erste Zeile
\[
    \begin{pmatrix}
        1 & a_{12}' & \dots & a_{1n} & b_{1}'
    \end{pmatrix}
    \text{ mit }
    a_{1k}' = \frac{a_{1k}}{a_{11}}
.\]
Für $i=2,\dots, m$ addiere das ($-a_{i1}$)-fache der 1. Zeile zur $i$-ten
Zeile
\[
    (A,b) \rightsquigarrow (A', b')=\begin{pmatrix}
        1 & a_{12}' & \dots & a_{1n}' & b_1'\\
        0 & a_{22}' & \dots & a_{2n}' & b_2'\\
        &&\vdots\\
        0 & a_{m2}' & \dots & a_{mn}' & b_n'
    \end{pmatrix} = \begin{pmatrix}
        1 & a_{12}' & \dots & a_{1n}' & b_1'\\
        0\\
        \vdots & \multicolumn{4}{c}{B} 
    \end{pmatrix}
.\]
Ist $B=0\implies \mathrm{rg}\ A=1$ und die Halbdiagonalen-Form ist erreicht.
Sonst erhält man durch Vertauschen von Zeilen und Spalten, dass $a_{22}'\neq 0$.
Wiederhole die Schritte sodass
\[
    (A', b') \rightsquigarrow (A'',b'') = \begin{pmatrix}
        1 & a_{12}' & a_{13}' & \dots & a_{1n}' & b_1'\\
        0 & 1 & a_{23}' & \dots & a_{2n}' & b_2'\\
        0 & 0 & a_{33}' & \dots & a_{3n}' & b_3'\\
        &\vdots&&&\vdots\\
        0 & 0 & a_{m3}' & \dots & a_{mn}' & b_n'
    \end{pmatrix}     
.\]
Wiederhole das ganze Verfahren bis die \textbf{Matrix in Halbdiagonalen-Form} ist.
\[
    \begin{pmatrix}
        1 & \tilde{a}_{12} & \dots & \tilde{a}_{1r} & \tilde{a}_{1,r+1} & \dots & \tilde{a}_{1n} & \tilde{b}_1\\
        0 & 1 & \dots & \tilde{a}_{2r} & \tilde{a}_{2,r+1} & \dots & \tilde{a}_{2n} & \tilde{b}_2\\
        \vdots & \vdots & \ddots & \vdots  & & \vdots\\
        0 & 0 & 0 & 1 & \tilde{a}_{r,r+1} & \dots & \tilde{a}_{rn} & \tilde{b}_r\\
        \hline\\
        0 & 0 & 0 & 0 & 0 & \dots & 0 & \tilde{b}_{r+1}\\
        &&&& \vdots\\
        0 & 0 & 0 & 0 & 0 & \dots & 0 & \tilde{b}_{m}
    \end{pmatrix}    
.\]
Falls für ein $b_k \in \{\tilde{b}_{r+1}, \dots, \tilde{b}_{m}\}\ :\ b_k \neq 0$ gilt,
so hat das Gleichungssystem \textbf{keine Lösung}.


Addiere nun geeignet Vielfache der $r$-ten Zeile zu den Zeilen darüber, sodass
in der $r$-ten Spalte in den oberen ($r-1$)-Zeilen nur 0 steht. Mach gleiche
für die $r-1$-ten Zeile usw. bis die \textbf{Gauß-Normalform} erreicht ist
\[
    (\hat{A}, \hat{b}) = \begin{pmatrix}
        1 & \dots & 0 & \mid & \hat{a}_{r,r+1} & \dots & \hat{a}_{1n} &\mid & \hat{b}_1\\
        & \ddots & & \mid & & \vdots & & \mid\\
        0 & \dots & 1 & \mid & \hat{a}_{r,r+1} & \dots &  \hat{a}_{rn} & \mid & \hat{b}_r\\
        \hline\\
        \multicolumn{9}{c}{\vdots}\\
        \multicolumn{9}{c}{0}\\
        \multicolumn{9}{c}{\vdots}
    \end{pmatrix} 
.\]
Seien nun folgende Vektoren definiert
\[
    \hat{b}:=\begin{pmatrix}
        \hat{b}_1\\
        \vdots\\
        \hat{b}_r\\
        0\\
        \vdots\\
        0
    \end{pmatrix},
    \mu_1 := \begin{pmatrix}
        -\hat{a}_{1,r+1}\\
        \vdots\\
        -\hat{a}_{r,r+1}\\
        1\\
        0\\
        \vdots\\
        0
    \end{pmatrix},\dots,
    \mu_{n-r} := \begin{pmatrix}
        -\hat{a}_{1n}\\
        \vdots\\
        -\hat{a}_{rm}\\
        0\\
        \vdots\\
        0\\
        1
    \end{pmatrix}
.\]
\begin{theorem}[Lösung des Gleichungssystems]
    Sei $Ax = b$ ein lineares Gleichungssystem mit $A\in M(n\times n), b\in \mathbb{R}^m$.
    Dann gilt
    \[
        \text{Lös}(A,b)=\{ x \in \mathbb{R}^m\ \mid\ x = \hat{b} + \lambda_1\mu_1 + \dots \lambda_{n-r}\mu_{n-r}\ \text{mit}\ \lambda_1,\dots,\lambda_{r-n} \in \mathbb{R} \}    
    .\]
\end{theorem}
    \pagebreak
    \section{Frage 6}
\textbf{TODO: Meint er nur ein Beispiel oder Lösung mittels Gauß?}\\
\textit{Beispiele zu linearen Gleichungssystemen.}
\begin{example}
    Lineares Gleichungssystem als Gleichungen.
    \begin{align*}
        2x_1 + x_2 - 2x_3 + 3x_4 &= 4\\
        3x_1 + 2x_2 - x_3 + 2x_4 &= 6\\
        3x_1 + 3x_2 + 3x_3- 3x_4 &= 6
    \end{align*}
\end{example}
    \pagebreak
    \section{Frage 7}
\textit{Formulieren Sie genau die Definition einer Basis.}

\begin{definition}[Basis]
    Sei $V$ ein Vektorraum. Das $n$-Tupel ($x_1, \dots, x_n$) von Vektoren aus $V$ heißt
    (geordnete) Basis von $V$, wenn gilt:
    \begin{enumerate}
        \item $x_1,\dots,x_n$ sind linear unabhängig
        \item $\mathrm{lin}\{x_1,\dots,x_n\} = V$ (lineare Hülle)
    \end{enumerate}
\end{definition}
\begin{definition}[linear abhängig]
    Die Vektoren $x_1,\dots,x_n$ heißen \underline{linear} \underline{abhängig}, wenn es Skalare
    $\lambda_1,\dots,\lambda_n\in \mathbb{K}$ gibt, die \textbf{nicht alle} $0$ sind, sodass
    \[
        0 = \lambda_1 x_1 + \dots + \lambda_n x_n.
    \]
    Also: $0$ lässt sich als nicht-triviale Linearkombination von $x_1,\dots,x_n$ dar\-stellen.

    Die Vektoren heißen außerdem \underline{linear unabhängig}, wenn sie nicht linear abhängig
    sind.
\end{definition}
\begin{definition}[lineare Hülle]
    Die Menge aller Linearkombinationen von $x_1,\dots,x_n$
    \[
        \mathrm{lin}\{x_1,\dots,x_n\} := \{
            x = \lambda_1 x_1 + \dots + \lambda_n x_n\ \mid\ \lambda_1,\dots,\lambda_n\in \mathbb{K}
        \}
    \]
    heißt die lineare Hülle von $x_1,\dots,x_n$.
\end{definition}
    \pagebreak
    \section{Frage 8}
\textit{Formulieren Sie die wichtigsten Sätze zur Basis.}

\begin{theorem}
    ($x_1,\dots,x_n$) ist eine Basis von $V$ $\Longleftrightarrow$ 
    $\forall x \in V$ gibt es \underline{eindeutig} bestimmte 
    $\lambda_1,\dots,\lambda_n\in\mathbb{K}$ mit 
    \[
        x = \lambda_1 x_1 + \dots + \lambda_n x_n    
    .\]
\end{theorem}
\begin{theorem}[Invarianz der Basislänge]
    Besitzt ein Vektorraum $V$ eine Basis, dann haben alle Basen von $V$
    gleichviele Elemente.
\end{theorem}
\begin{theorem}[Basisauswahlsatz]
    Sei $V$ endlich erzeugt, also $V=\mathrm{lin}\{v_1,\dots,v_n\}$.
    Dann erhält man durch Weglassen geeigneter Elemente von 
    $\{v_1,\dots,v_n\}$ eine Basis von $V$.
\end{theorem}
\begin{theorem}[Basisergänzungssatz]
    Sei $V$ endlich erzeugt und seien $v_1,\dots,v_k$ linear unabhängige
    Vektoren von $V$.\\
    Dann lassen sich $v_1,\dots,v_k$ zu einer Basis ergänzen, d.h. es gibt\\ 
    $v_{k+1},\dots,v_{k+e}\ (e\geq 0)$, sodass $v_1,\dots,v_k,v_{k+1},\dots,v_{k+e}$
    eine Basis von $V$ bildet.
\end{theorem}
\begin{corollary}
    Jeder endlich erzeugte Vektorraum $V$ besitzt eine Basis.
\end{corollary}
\begin{corollary}
    Sei $V$ ein Vektorraum und $\mathrm{dim}\ V=k$. Seien weiters $v_1,\dots,v_k\in V$.
    Dann sind folgende Aussagen \underline{äquivalent}:
    \begin{enumerate}
        \item ($v_1,\dots,v_k$) ist eine Basis von $V$,
        \item $\mathrm{lin}\{v_1,\dots,v_k\} = V$,
        \item $v_1,\dots,v_k$ sind linear unabhängig.
    \end{enumerate}
\end{corollary}
\begin{corollary}
    Sei $V$ ein Vektorraum mit $\mathrm{dim}\ V=k$. Sei weiters $U$ ein Teilraum
    von $V$. Dann sind folgende Aussagen äquivalent:
    \begin{enumerate}
        \item $U=V$
        \item $\mathrm{dim}\ U=\mathrm{dim}\ V$
    \end{enumerate}
\end{corollary}
    \pagebreak
    \section{Frage 9}
\textit{Die Dimensionsformel für lineare Abbildungen: Formulierung und Beweis.}
\begin{theorem}[Dimensionsformel für lineare Abbildungen]
    Seien $V,W$ Vektorräume und $V$ endlich-dimensional. Sei $T:V\to W$
    eine lineare Abbildung. Dann folgt
    \[
        \mathrm{dim} (\mathrm{ker}\ T) + \mathrm{dim}(\mathrm{im}\ T) = \mathrm{dim}\ V
    .\]
\end{theorem}
\begin{proof}
    Sei $V$ endlich-dimensional mit $\mathrm{dim}\ V=n$, dann ist $\mathrm{ker}\ T$ ein 
    Teilraum von $V$, wodurch auch $\mathrm{ker}\ T$ endlich-dimensional ist. Sei
    weiters $v_1,\dots,v_k$ eine Basis von $\mathrm{ker}\ T$.
    Nach dem Basisergänzungssatz kann diese zu einer Basis $v_1,\dots,v_k,v_{k+1},\dots,v_n$
    von $V$ ergänzt werden.\\
    \textit{Behauptung}: $T(v_{k+1}),\dots,T(v_n)$ ist eine Basis von $\mathrm{im}\ T$.\\
    \begin{enumerate}
        \item[(1)] $\mathrm{lin}\{T(v_{k+1},\dots,T(v_{n}))\} = \mathrm{im}\ T$\\
        Sei $x\in V$. Da $v_1,\dots,v_n$ eine Basis von $V$ ist, gibt es 
        eindeutig bestimmte $\lambda_1,\dots,\lambda_n\in \mathbb{K}$, sodass
        \[
            x = \lambda_1v_1 + \dots + \lambda_n v_n    
        \]
        \begin{align*}
            \implies T(x) &= T(\lambda_1v_1 + \dots + \lambda_n v_n)\\
            &= \lambda_1 \underbrace{T(v_1)}_{=0} + \dots 
            + \lambda_k \underbrace{T(v_k)}_{=0} +
            \lambda_{k+1} T(v_{k+1}) + \dots +
            \lambda_n T(v_n)\\
            &=\lambda_{k+1} T(v_{k+1}) + \dots + \lambda_n T(v_n) 
            \in \mathrm{lin}\{T(v_{k+1},\dots,T(v_n))\}\\
            \implies \mathrm{im}\ T &\subseteq \mathrm{lin}\{
                T(v_{k+1},\dots,T(v_n))
            \}\\
            &\text{Klarerweise ist auch } \mathrm{lin}\{T(v_{k+1},\dots,T(v_n))\}\subseteq \mathrm{im}\ T\\
            \implies \mathrm{im}\ T &= \mathrm{lin}\{T(v_{k+1},\dots,T(v_n))\}
        \end{align*}
        \item[(2)] $T(v_{k+1},\dots,T(v_n))$ linear unabhängig:\\
        Sei $\lambda_{k+1}T(v_{k+1}) + \dots + \lambda_n T(v_n) = 0$
        \begin{align*}
            \implies &T(\lambda_{k+1}v_{k+1}+\dots +\lambda_nv_n) = 0\\
            \implies &\lambda_{k+1}v_{k+1} + \dots + \lambda_n v_n \in \mathrm{ker}\ T
        \end{align*}
        Da $v_1,\dots,v_k$ eine Basis von $\mathrm{ker}\ $T ist, gibt es 
        $\lambda_1,\dots,\lambda_n\in \mathbb{K}$ mit\\
        $\lambda_{k+1}v_{k+1} + \dots + \lambda_nv_n = \lambda_1 v_1 + \dots + \lambda_k v_k$.
        \[
            \implies \lambda_1v_1 + \dots + \lambda_kv_k - \lambda_{k+1}v_{k+1} - \dots - \lambda_n v_n = 0    
        .\]
        Da $v_1,\dots,v_n$ linear unabhängig sind, folgt
        \[
            \lambda_1 = \dots = \lambda_k = \lambda_{k+1} = \dots = \lambda_n = 0   
        .\]
        Also speziell $\lambda_{k+1} = \dots = \lambda_n = 0$
        \[
            \implies T(v_{k+1}),\dots,T(v_n)\ \text{sind l.u.}
        \]
        \item[(3)] $T(v_{k+1}),\dots,T(v_n)$ ist eine Basis von $\mathrm{im}\ T$
        \[
            \implies \mathrm{dim}(\mathrm{im}\ T) = n-k = \mathrm{dim}\ V - \mathrm{dim}(\mathrm{ker}\ T)    
        .\]
    \end{enumerate} 
\end{proof}
    \pagebreak
    \section{Frage 10}
\textit{Schnitt Gerade - Hyperebene: Beispiele}
    \pagebreak
    \section{Frage 11}
\textit{Schnitt von zwei affinen Teilräumen: Beispiele}
\begin{gather*}
    L\ :\ x =\begin{pmatrix}
        2\\1\\3\\0
    \end{pmatrix} + \lambda_1\begin{pmatrix}
        1\\2\\1\\0
    \end{pmatrix} + \lambda_2 \begin{pmatrix}
        0\\1\\1\\2
    \end{pmatrix}\qquad\text{Ebene im }\mathbb{R}^4\\
    H\ :\ x_1 + x_2 + x_3 + x_4 = 5\qquad\text{Hyperebene}
\end{gather*}
\underline{Schritt 1}: Bestimme die transponierte Matrix $C^T$ aus alle Vektoren von $L$ (außer dem 1.) und bringe sie auf die Gauß-Normalform.
\[
    C = \begin{pmatrix}
        1 & 0\\
        2 & 1\\
        1 & 1\\
        0 & 2
    \end{pmatrix} \implies C^T = \begin{pmatrix}
        1 & 2 & 1 & 0\\
        0 & 1 & 1 & 2
    \end{pmatrix}
\]
\begin{elimination}{4}{1.1em}{1.1}
    \eliminationstep
    {
        1 & 2 & 1 & 0\\
        0 & 1 & 1 & 2
    }
    {
        -(2)\mathrm{II} + \mathrm{I} \to \mathrm{I}
    }
    \eliminationstep
    {
        1 & 0 & -1 & -4\\
        0 & 1 & 1 & 2
    }
    {
        \\
    }
\end{elimination}
\underline{Schritt 2}: Aus der Gauß-Normalform die Lösungsvektoren $a_i^T$ auslesen und die transponierten Vektoren
in die Matrix $A$ schreiben.
\begin{gather*}
    a_1^T=\begin{pmatrix}
        1\\-1\\1\\0
    \end{pmatrix}, a_2^T =\begin{pmatrix}
        4\\-2\\0\\1
    \end{pmatrix}\\
    \implies A = \begin{pmatrix}
        a_1\\a_2
    \end{pmatrix}=\begin{pmatrix}
        1 & -1 & 1 & 0\\
        4 & -2 & 0 & 1
    \end{pmatrix}
\end{gather*}
\underline{Schritt 3}: Den inhomogenen Vektor $b$ berechnen, indem man $A$ mit dem 1. Vektor $p$ aus $L$ multipliziert.
\[
    b = Ap = \begin{pmatrix}
        1 & -1 & 1 & 0\\
        4 & -2 & 0 & 1
    \end{pmatrix}\begin{pmatrix}
        2\\ 1\\3\\0
    \end{pmatrix} = \begin{pmatrix}
        4\\6
    \end{pmatrix}
\]
\underline{Schritt 4}: Das Gleichungssystem für $L$ aufstellen und die Gleichung der Hyperebene $H$ anhängen.
\begin{gather*}
    \sysdelim..
    \systeme{
        x_1 - x_2 + x_3 = 4,
        4x_1 - 2x_2 + x_4 = 6,
        x_1 + x_2 + x_3 + x_4 = 5
    }
\end{gather*}
\underline{Schritt 5}: Das Gleichungssystem mittel Gauß-Elimination lösen.
\begin{elimination}[6]{4}{1.1em}{1.1}
    \eliminationstep
    {
        1 & -1 & 1 & 0 & 4\\
        4 & -2 & 0 & 1 & 6\\
        1 & 1 & 1 & 1 & 5
    }
    {
        (-4)\text{I} + \text{II} \to \text{II}\\
        (-1)\text{I} + \text{III} \to \text{III}
    }
    \eliminationstep
    {
        1 & -1 & 1 & 0 & 4\\
        0 & 2 & -4 & 1 & -10\\
        0 & 2 & 0 & 1 & 1
    }
    {
        (-1)\text{II} + \text{III} \to \text{III}\\
        \nicefrac{1}{2}\cdot\text{III}
    }
    \\[10pt]
    \eliminationstep
    {
        1 & -1 & 1 & 0 & 4\\
        0 & 1 & -2 & \nicefrac{1}{2} & -5\\
        0 & 0 & 4 & 0 & 11
    }
    {
        \nicefrac{1}{4}\cdot \text{III}
    }
    \eliminationstep
    {
        1 & -1 & 1 & 0 & 4\\
        0 & 1 & -2 & \nicefrac{1}{2} & -5\\
        0 & 0 & 1 & 0 & \nicefrac{11}{4}
    }
    {
        \text{I} + \text{II} \to \text{I}
    }
    \\[10pt]
    \eliminationstep
    {
        1 & 0 & -1 & \nicefrac{1}{2} & -1\\
        0 & 1 & -2 & \nicefrac{1}{2} & -5\\
        0 & 0 & 1 & 0 & \nicefrac{11}{4}
    }
    {
        \text{I} + \text{III} \to \text{I}\\
        \text{II} + 2\cdot\text{III} \to \text{II}
    }
    \eliminationstep
    {
        1 & 0 & 0 & \nicefrac{1}{2} & \nicefrac{7}{4}\\
        0 & 1 & 0 & \nicefrac{1}{2} & \nicefrac{1}{2}\\
        0 & 0 & 1 & 0 & \nicefrac{11}{4}
    }
    {
        \\
    }
\end{elimination}
\[
    \implies x = \begin{pmatrix}
        \nicefrac{7}{4}\\
        \nicefrac{1}{2}\\
        \nicefrac{11}{4}\\
        0
    \end{pmatrix} + \lambda \begin{pmatrix}
        -\nicefrac{1}{2}\\
        -\nicefrac{1}{2}\\
        0\\
        1
    \end{pmatrix}
\]
Daraus folgt, dass $L\cap H$ eine Gerade durch den $\mathbb{R}^4$ ist.
    \pagebreak
    \section{Frage 12}
\textit{Die inverse Matrix: Definition, Berechnung der inversen Matrix und Beispiele.}
\begin{definition}[invertierbare Matrix]
    Eine $n\times n$-Matrix heißt \underline{invertierbar}, wenn
    \begin{align*}
        f_A\ :\ &\mathbb{R}^n \to \mathbb{R}^n\\
        & f_A(x) = Ax
    \end{align*}
    bijektiv ist.
\end{definition}
\begin{corollary}
    $A\in M(n\times n)$ ist invertierbar genau dann, wenn $\text{rg}(A)=n$.
\end{corollary}
\begin{definition}[Inverse Matrix]
    Sei $A$ eine $n\times n$-Matrix, dann heißt $A^{-1}$ die dazu inverse Matrix, wenn
    \[
        A^{-1}\cdot A = E
    \]
    gilt.
\end{definition}
\subsection*{Berechnung der inversen Matrix}
Suche die Spaltenvektoren $a_1,\dots,a_n$ von $A^{-1} = \begin{pmatrix}
    a_1 & \dots & a_n
\end{pmatrix}$. Da $AA^{-1}= E$ folgt $\forall i=1,\dots,n$
\[
    Aa_i=e_i
.\]
Da $\text{rg}(A) = n$, ist $Ax = e_i$ eindeutig lösbar $\forall i = 1,\dots, n$. Weiters ist $a_i$ die eindeutige
Lösung von 
\[
    Ax = e_i\qquad\forall i=1,\dots n.
\]
Dadurch entstehen $n$ Gleichungssysteme, die simultan gelöst werden können, indem man die Vektoren $e_1, \dots, e_n$
nebeneinander hinter die Matrix $A$ schreibt und dann den Gauß-Algorithmus anwendet.
\begin{multline*}    
    \left(
        \begin{array}{c | c c c c}
            A & e_1 & e_2 & \dots & e_n
        \end{array}
    \right) = \left(\begin{array}{c | c c c}
        \multirow{4}{*}{\Huge$A$} & 1 & & 0\\
        & & \ddots &\\
        & 0 & & 1
    \end{array}\right) \xrightarrow[]{\text{Gauß}}
    \left(
        \begin{array}{c c c | c c c}
            1 & & 0 & \multirow{3}{*}{$a_1$} & \multirow{3}{*}{$\dots$} & \multirow{3}{*}{$a_n$}\\
              & \ddots &\\
            0 & & 1
        \end{array}
    \right)\\
    =\left(
        \begin{array}{ccc|r}
            1 & & 0 & \multirow{4}{*}{\Huge$A^{-1}$}\\
              & \ddots &\\
            0 & & 1 &
        \end{array}
    \right)
\end{multline*}
\begin{example}
    Sei $A$ eine $4\times 4$-Matrix mit
    \[
        A = \begin{pmatrix}
            1 & 0 & 1 & 1\\
            1 & 1 & 2 & 1\\
            0 & -1 & 0 & 1\\
            1 & 0 & 0 & 2
        \end{pmatrix}
    \]
    
    \begin{elimination}[4]{4}{0.8em}{1.1}
        \eliminationstep
        {
            1 & 0 & 1 & 1 & 1 & 0 & 0 & 0\\
            1 & 1 & 2 & 1 & 0 & 1 & 0 & 0\\
            0 & -1 & 0 & 1 & 0 & 0 & 1 & 0\\
            1 & 0 & 0 & 2 & 0 & 0 & 0 & 1
        }
        {
            \text{II} - \text{I} \to \text{II}\\
            \text{IV} - \text{I} \to \text{IV}
        }
        \eliminationstep
        {
            1 & 0 & 1 & 1 & 1 & 0 & 0 & 0\\
            0 & 1 & 1 & 0 & -1 & 1 & 0 & 0\\
            0 & -1 & 0 & 1 & 0 & 0 & 1 & 0\\
            0 & 0 & -1 & -1 & -1 & 0 & 0 & 1
        }
        {
            \text{II} + \text{III} \to \text{III}
        }
        \\[10pt]
        \eliminationstep
        {
            1 & 0 & 1 & 1 & 1 & 0 & 0 & 0\\
            0 & 1 & 1 & 0 & -1 & 1 & 0 & 0\\
            0 & 0 & 1 & 1 & -1 & 1 & 1 & 0\\
            0 & 0 & -1 & 1 & -1 & 0 & 0 & 1
        }
        {
            \text{III} + \text{IV} \to \text{IV}
        }
        \eliminationstep
        {
            1 & 0 & 1 & 1 & 1 & 0 & 0 & 0\\
            0 & 1 & 1 & 0 & -1 & 1 & 0 & 0\\
            0 & 0 & 1 & 1 & -1 & 1 & 1 & 0\\
            0 & 0 & 0 & 2 & -2 & 1 & 1 & 0
        }
        {
            \text{I} - \text{III} \to \text{I}\\
            \text{II} - \text{III} \to \text{II}\\
            \nicefrac{1}{2}\cdot \text{IV}
        }
        \\[10pt]
        \eliminationstep
        {
            1 & 0 & 0 & 0 & 2 & -1 & -1 & 0\\
            0 & 1 & 0 & -1 & 0 & 0 & -1 & 0\\
            0 & 0 & 1 & 1 & -1 & 1 & 1 & 0\\
            0 & 0 & 0 & 1 & -1 & \frac{1}{2} & \frac{1}{2} & \frac{1}{2}
        }
        {
            \text{II} + \text{IV} \to \text{II}\\
            \text{III} - \text{IV} \to \text{III}
        }
        \eliminationstep
        {
            1 & 0 & 0 & 0 & 2 & -1 & -1 & 0\\
            0 & 1 & 0 & 0 & -1 & \frac{1}{2} & -\frac{1}{2} & \frac{1}{2}\\
            0 & 0 & 1 & 0 & 0 & \frac{1}{2} & \frac{1}{2} & -\frac{1}{2}\\
            0 & 0 & 0 & 1 & -1 & \frac{1}{2} & \frac{1}{2} & \frac{1}{2}
        }
        {
            \\
        }
    \end{elimination}
    \[
        \implies A^{-1} = \begin{pmatrix}
            2 & -1 & 1 & 0\\
            -1 & \nicefrac{1}{2} & -\nicefrac{1}{2} & \nicefrac{1}{2}\\
            0 & \nicefrac{1}{2} & \nicefrac{1}{2} & -\nicefrac{1}{2}\\
            -1 & \nicefrac{1}{2} & \nicefrac{1}{2} & \nicefrac{1}{2}
        \end{pmatrix}
    \]
\end{example}
    \pagebreak
    \section{Frage 13}
\textit{Die Ungleichung von Cauchy-Schwarz: Formulierung und Beweis.}
\begin{theorem}[Ungleichung Cauchy-Schwarz]
    Sei $V$ ein euklidischer Vektorraum. Dann gilt $\forall~x,y\in~V$
    \[
    |\langle x,y \rangle|\leq \sqrt{\langle x,x\rangle}\cdot\sqrt{\langle y,y\rangle}\]
\end{theorem}
\begin{proof} Cauchy-Schwarz Ungleichung\\
    Fall 1: Sei $\langle y,y\rangle = 0$.
    \begin{align*}
        &\implies y = 0\\
        &\implies \langle x,0\rangle = \langle x, 0\cdot 0\rangle = 0\cdot \langle x, 0\rangle = 0\\
        &\implies 0 \leq 0 \implies \text{Cauchy-Schwarz Ungleichung}
    \end{align*}
    Fall 2: Sei etwa $\langle y,y\rangle\neq 0$ (sonst $x$ und $y$ tauschen). Sei $z$ ein weitere Vektor, 
    der wie folgt definiert ist.
    \[
    z := x - \frac{\langle x, y\rangle}{\langle y,y\rangle}\cdot y\]
    \begin{align*}
        \implies 0 &\leq \langle z, z\rangle = \left\langle x - \frac{\langle x, y\rangle}{\langle y,y\rangle}\cdot y, x - \frac{\langle x, y\rangle}{\langle y,y\rangle}\cdot y \right\rangle\\
        &= \langle x,x\rangle - \frac{\langle x, y\rangle}{\langle y,y\rangle}\langle y,x\rangle - \frac{\overline{\langle x,y\rangle}}{\langle y, y\rangle}\langle x,y\rangle + \frac{\langle x,y\rangle}{\langle y,y\rangle}\frac{\overline{\langle x,y\rangle}}{\langle y,y\rangle}\langle y,y\rangle\\
        &= \langle x,x\rangle - \frac{\langle x,y\rangle \overline{\langle x,y\rangle}}{\langle y,y\rangle} - \frac{\overline{\langle x,y\rangle}\langle x,y\rangle}{\langle y,y \rangle} + \frac{\langle x,y\rangle \overline{\langle x,y\rangle}}{\langle y,y\rangle}\\
        &= \langle x,x\rangle - \frac{\langle x,y\rangle \overline{\langle x,y\rangle}}{\langle y,y\rangle} = \langle x,x\rangle - \frac{|\langle x,y\rangle|^2}{\langle y,y\rangle}\\
        \Longleftrightarrow 0 &\leq \langle x,x\rangle - \frac{|\langle x,y\rangle|^2}{\langle y,y\rangle}\qquad\mid\cdot \langle y,y\rangle, +|\langle x,y\rangle|^2\\
        \Longleftrightarrow |\langle x,y\rangle|^2 &\leq \langle x, x\rangle\langle y,y\rangle\qquad\mid\sqrt[]{\ }\\
        \Longleftrightarrow \langle x,y\rangle &\leq \sqrt{\langle x, x\rangle}\cdot \sqrt{\langle y,y\rangle}
    \end{align*}
\end{proof}
    \pagebreak
    \section{Frage 14}
\textit{Die Dreiecksungleichung für die Norm in einem euklidischen Vektorraum: 
Formulierung und Beweis.}

\begin{definition}[Dreiecksungleichung der Norm]
    Sei $\norm{\ .\ } : V \to \mathbb{R}$ eine Norm, dann gilt per dessen Definition
    \[
        \norm{x + y} \leq \norm{x} + \norm{y}\qquad\forall x,y\in V
    .\]
\end{definition}
\begin{proof} Dreiecksungleichung der Norm\\
    Sei $z = a+ib=\textrm{Re}(z) + \textrm{Im}(z)\in \mathbb{C} \implies \vert\mathrm{Re}(z)\vert \leq \vert z\vert$
    \begin{align*}
        \implies \norm{x+y}^2 &= \inner{x+y}{x+y}\\
        &= \inner{x}{x} + \inner{x}{y} + \inner{y}{x} + \inner{y}{y}\\
        &= \norm{x}^2 + \underbrace{\inner{x}{y} + \overline{\inner{x}{y}}}_{\text{Re}(z) = (z + \overline{z}) / 2} + \norm{y}^2\\
        &= \norm{x}^2 + 2\textrm{Re}(\inner{x}{y}) + \norm{y}^2\\
        &\leq \norm{x}^2 + 2\vert \textrm{Re}(\inner{x}{y}) \vert + \norm{y}^2\\
        &\leq \norm{x}^2 + 2\vert \inner{x}{y}\vert + \norm{y}^2\\
        \text{Cauchy-Schwarz} &\leq \norm{x}^2 + 2 \sqrt{\inner{x}{x}}\sqrt{\inner{y}{y}} + \norm{y}^2\\
        &=\norm{x}^2 + 2 \norm{x}\norm{y} + \norm{y}^2\\
        &= (\norm{x} + \norm{y})^2\\
        \implies \norm{x + y} &\leq \norm{x} + \norm{y}\qquad\forall x,y\in V
    \end{align*}
\end{proof}
    \pagebreak
    \section{Frage 15}
\textit{Orthonormalbasis: Genaue Definition: Beschreiben Sie die äquivalenten 
Formulierung einer Orthonormalbasis.}
\begin{definition}
    Sei $V$ ein euklidischer Vektorraum und seien $v_1,\dots,v_n\in V$.
    \begin{enumerate}
        \item[(1)] $\{v_1,\dots,v_n\}$ heißt \underline{Orthogonalsystem (OGS)}, wenn
        \[\inner{v_i}{v_j} = 0\qquad\forall i,j=1,\dots,n, i\neq j\] 
        \item[(2)] $\{v_1,\dots,v_n\}$ heißt \underline{Orthonormalsystem (ONS)}, wenn
        \[
            \inner{v_i}{v_j} = \begin{cases}
                1 &i = j\\
                0 &i \neq j
            \end{cases}\qquad\forall i,j =1,\dots,n
        .\]
        Also: $\{v_1,\dots,v_n\}$ ist OGS mit $\norm{v_i}=1, \forall i = 1,\dots,n$.
        \item[(3)] $\{v_1,\dots,v_n\}$ heißt \underline{Orthonormalbasis (ONB)}, wenn
        \begin{enumerate}
            \item $\{v_1,\dots,v_n\}$ ist ein Orthonormalsystem
            \item $\{v_1,\dots,v_n\}$ ist eine Basis von $V$
        \end{enumerate}
    \end{enumerate}
\end{definition}
\begin{corollary}
    $\{v_1,\dots,v_n\}$ ist eine Orthonormalbasis genau dann, wenn $\{v_1,\dots,v_n\}$
    ein Orthonormalsystem ist und $\mathrm{lin}\{v_1,\dots,v_n\}=V$.
\end{corollary}
\begin{theorem}[Fourier Entwicklung]
    Sei $V$ ein euklidischer Vektorraum und $\{v_1,\dots,v_n\}$ ein Orthonormalsystem 
    von $V$. Dann sind folgende Aussagen äquivalent:
    \begin{enumerate}
        \item[(1)] $\{v_1,\dots,v_n\}$ ist eine Orthonormalbasis von $V$
        \item[(2)] $\forall x\in V$ gilt:
        \[
            x = \sum_{i = 1}^{n}\inner{x}{v_i}v_i\qquad\text{Fourierdarstellung}  
        \]
        \item[(3)] $\forall x,y\in V$ gilt:
        \[
            \inner{x}{y} = \sum_{i=1}^{n}\inner{x}{v_i}\overline{\inner{y}{v_i}}
            \qquad\text{Parseval'sche Gleichung}   
        .\]
        \item[(4)] $\forall x \in V$ gilt:
        \[
            \norm{x}^2 = \sum_{i = 1}^{n}\vert \inner{x}{v_i}\vert^2
            \qquad \text{Parseval'sche Gleichung}    
        .\]
        \item[(5)] Ist $x\in V$ mit $\inner{x}{v_i} = 0$ für alle $i=1,\dots,n$, so gilt: $x=0$. 
    \end{enumerate}
\end{theorem}
\end{document}