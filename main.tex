\documentclass{article}
\usepackage[T1]{fontenc}
\usepackage{babel}[ngerman]
\usepackage{hyperref}

\setcounter{secnumdepth}{0}

\usepackage{hyphenat}
\hyphenation{Mathe-matik wieder-gewinnen}

\usepackage{amssymb, amsmath, amsthm}
\usepackage{tcolorbox}
\usepackage{nicefrac}

\usepackage{tcolorbox}
\tcbuselibrary{theorems,skins,breakable}

\definecolor{royalazure}{rgb}{0.0, 0.22, 0.66}

\tcolorboxenvironment{theorem}{
enhanced jigsaw,colframe=royalazure,interior hidden,
breakable,before skip=10pt,after skip=10pt }

\tcolorboxenvironment{definition}{
enhanced jigsaw,colframe=cyan,interior hidden,
breakable,before skip=10pt,after skip=10pt }

\theoremstyle{definition}
\newtheorem*{definition}{Definition}

\theoremstyle{definition}
\newtheorem*{theorem}{Satz}

\theoremstyle{remark}
\newtheorem*{example}{Beispiel}

\theoremstyle{remark}
\newtheorem*{corollary}{Korollar}

\title{Fragenkatalog Ausarbeitung\\
\large Lineare Algebra 2024}

\begin{document}
    \maketitle
    \tableofcontents
    \pagebreak
    \section{Frage 3}
\textit{Das Produkt von Matrizen: Definition und Motivation durch
lineare Abbildung.}

\begin{definition}[Matrix-Multiplikation]
    Sei $A\in M(m\times k), B\in M(k\times n)$.
    \[
        A\cdot B = (c_{ij})\in M(m\times n)  
    \]
    mit
    \[
        c_{ij} :=a_{i1}b_{1j} + a_{i2}b_{2j}+ \dots a_{ik}b_{kj}
    = \sum_{l=1}^{k}a_{il}b_{lj}
    \]
\end{definition}

\textbf{Motivation durch lineare Abbildung}. Seien $A\in M(m\times k)$,
$B\in~M(k\times~n)$ Matrizen und $f_A, f_B$ die entsprechenden 
linearen Abbildungen. Wir definieren $A\cdot B$ so, dass
\[
    f_{AB}=f_A\circ f_B
.\]
Sei $e_j$ der $j$-te Einheitsvektor:
\[
    e_j=\begin{pmatrix}
        0\\ \vdots\\ 1\\ \vdots\\ 0
    \end{pmatrix}
.\]
Setzt man $e_j$ in $f_{AB}$ ein, so erhält man
\begin{align*}
    f_A \circ f_B (e_j) &= f_A(f_B(e_j)) = f_A(Be_j)\\
    &=f_A=\begin{pmatrix}
        b_{1j}\\ \vdots\\ b_{kj}
    \end{pmatrix} = A \begin{pmatrix}
        b_{1j}\\ \vdots\\ b_{kj}
    \end{pmatrix} = \begin{pmatrix}
        a_{11}b_{1j} + \dots + a_{1k}b_{kj}\\
        \vdots\\
        a_{i1}b_{1j} + \dots + a_{ik}b_{kj}\\
        \vdots\\
        a_{m1}b_{1j} + \dots + a_{mk}b_{kj}
    \end{pmatrix}\\
    &= f_{AB}(e_j) = j\text{-te Spalte der Matrix }AB
.\end{align*}
    \pagebreak
    \section{Frage 4}
\textit{Wie berechnet man die Anzahl der Wege zwischen zwei Ecken in
einem gerichteten Graphen? Formulierung und beweisen Sie den entsprechenden
Satz.}

\begin{theorem}[Anzahl der Wege mit Länge $k$]
    Sei $G=(V,E)$ ein gerichteter Graph mit Adjazenten-Matrix $A$. Dann ist
    die Anzahl der Wege von $x_i$ nach $x_j$ ($x_i,x_j\in V$) der Länge $k$
    der Koeffizient der Matrix $A^k=A\cdot A \cdots$ an der Stelle $(i,j)$.
\end{theorem}

\begin{proof}
    Anzahl der Wege mit Länge $k$.

    Beweis über Induktion nach $k$.\\
    \underline{$k = 1$} Gilt aufgrund der Definition der Adjazenten-Matrix.\\
    \underline{$k \to k + 1$} Angenommen der Satz gilt für $k$. Wir betrachten
    einen Weg der Länge $k+1$ von $x_i$ nach $x_j$
    \[
        x_i \to \underbrace{x_e \to \cdots \to x_j}_{\text{Weg der Länge }k}  
    \]
    für ein $x_e$, also $a_{ie} = 1$. Sei $A^k=(b_{pq})\in M(n\times n)$.
    Nach der Induktionshypothese gibt es $b_{ej}$ Wege der Länge $k$ von 
    $x_e$ nach $x_j$, also $b_{ej}$ Wege der Länge $k+1$ von $x_i$ 
    über $x_e$ nach $x_j$. Wir betrachten alle möglichen Wege über $x_e$ für
    $e=1,\dots,n$.

    Es gibt $\sum_{e=1}^{n}a_{ie}b_{ej}$ Wege der Länge $k+1$ von $x_i$ nach
    $x_j$. $\sum_{e=1}^{n}a_{ie}b_{ej}$ ist aber der Koeffizient von
    $A\cdot A^k=A^{k+1}$ an der Stelle $(i,j)$. Das heißt die Anzahl der Wege
    von $x_i$ nach $x_j$ der Länge $k+1$ ist der Koeffizient der Matrix
    $A^{k+1}$ an der Stelle $(i,j)$.
\end{proof}
    \pagebreak
    \section{Frage 5}
\textit{Der Gauß-Algorithmus: Beschreiben Sie genau wie man eine
Matrix auf die Gauß-Normalform bringt. Wie schaut dann die Lösung aus?}

\begin{definition}[Elementare Zeilen (Spalten) Umformungen]
    Sei\\ $A\in M(m\times~n)$. Es gibt 3 Typen von elementaren
    Zeilen (Spalten) Umformungen.
    \begin{enumerate}
        \item[(I)] Vertauschen von zwei Zeilen (Spalten).
        \item[(II)] Multiplikation einer Zeile (Spalte) mit einer Zahl 
        $\lambda\neq 0$. 
        \item[(III)] Addieren eines beliebigen Vielfachen einer Zeile (Spalte)
        zu einer anderen Zeile (Spalte).
    \end{enumerate}
\end{definition}

Sei $Ax=b$ ein Lineares Gleichungssystem mit der Form
\[
    \begin{pmatrix}
        a_{11} & \cdots & a_{1n}\\
        \vdots & \ddots & \vdots\\
        a_{m1} & \cdots & a_{mn}
    \end{pmatrix} \begin{pmatrix}
        x_1\\ \vdots\\ x_n
    \end{pmatrix} = \begin{pmatrix}
        b_1\\ \vdots\\ b_n
    \end{pmatrix}
.\]
Zuerst muss die Matrix $A$ um den inhomogenen Vektor $b$ erweitert werden
\[
    (A,b):= \begin{pmatrix}
        a_{11} & \cdots & a_{1n} & \mid & b_1\\
        \vdots & \ddots & \vdots & \mid & \vdots\\
        a_{m1} & \cdots & a_{mn} & \mid & b_n
    \end{pmatrix}
.\]
Anschließend muss die Matrix durch elementare Umformungen in 
\textbf{Halbdiagonalen Form} gebracht werden, wobei die Spalte
des inhomogenen Vektors nicht vertauscht werden darf.

\underline{Halbdiagonalen-Form}: Sei $A\neq0 \implies \exists a_{ik} \neq 0$.
Dann durch Zeilen- und Spalten-Vertauschen $a_{ik}$ an die Stelle $(1,1)$ 
bringen. Es kann also angenommen werden, dass $a_{11} \neq 0$. 

Multipliziere die erste Zeile mit $\displaystyle\frac{1}{a_{11}}$. 
Dadurch lautet die erste Zeile
\[
    \begin{pmatrix}
        1 & a_{12}' & \dots & a_{1n} & b_{1}'
    \end{pmatrix}
    \text{ mit }
    a_{1k}' = \frac{a_{1k}}{a_{11}}
.\]
Für $i=2,\dots, m$ addiere das ($-a_{i1}$)-fache der 1. Zeile zur $i$-ten
Zeile
\[
    (A,b) \rightsquigarrow (A', b')=\begin{pmatrix}
        1 & a_{12}' & \dots & a_{1n}' & b_1'\\
        0 & a_{22}' & \dots & a_{2n}' & b_2'\\
        &&\vdots\\
        0 & a_{m2}' & \dots & a_{mn}' & b_n'
    \end{pmatrix} = \begin{pmatrix}
        1 & a_{12}' & \dots & a_{1n}' & b_1'\\
        0\\
        \vdots & \multicolumn{4}{c}{B} 
    \end{pmatrix}
.\]
Ist $B=0\implies \mathrm{rg}\ A=1$ und die Halbdiagonalen-Form ist erreicht.
Sonst erhält man durch Vertauschen von Zeilen und Spalten, dass $a_{22}'\neq 0$.
Wiederhole die Schritte sodass
\[
    (A', b') \rightsquigarrow (A'',b'') = \begin{pmatrix}
        1 & a_{12}' & a_{13}' & \dots & a_{1n}' & b_1'\\
        0 & 1 & a_{23}' & \dots & a_{2n}' & b_2'\\
        0 & 0 & a_{33}' & \dots & a_{3n}' & b_3'\\
        &\vdots&&&\vdots\\
        0 & 0 & a_{m3}' & \dots & a_{mn}' & b_n'
    \end{pmatrix}     
.\]
Wiederhole das ganze Verfahren bis die \textbf{Matrix in Halbdiagonalen-Form} ist.
\[
    \begin{pmatrix}
        1 & \tilde{a}_{12} & \dots & \tilde{a}_{1r} & \tilde{a}_{1,r+1} & \dots & \tilde{a}_{1n} & \tilde{b}_1\\
        0 & 1 & \dots & \tilde{a}_{2r} & \tilde{a}_{2,r+1} & \dots & \tilde{a}_{2n} & \tilde{b}_2\\
        \vdots & \vdots & \ddots & \vdots  & & \vdots\\
        0 & 0 & 0 & 1 & \tilde{a}_{r,r+1} & \dots & \tilde{a}_{rn} & \tilde{b}_r\\
        \hline\\
        0 & 0 & 0 & 0 & 0 & \dots & 0 & \tilde{b}_{r+1}\\
        &&&& \vdots\\
        0 & 0 & 0 & 0 & 0 & \dots & 0 & \tilde{b}_{m}
    \end{pmatrix}    
.\]
Falls für ein $b_k \in \{\tilde{b}_{r+1}, \dots, \tilde{b}_{m}\}\ :\ b_k \neq 0$ gilt,
so hat das Gleichungssystem \textbf{keine Lösung}.


Addiere nun geeignet Vielfache der $r$-ten Zeile zu den Zeilen darüber, sodass
in der $r$-ten Spalte in den oberen ($r-1$)-Zeilen nur 0 steht. Mach gleiche
für die $r-1$-ten Zeile usw. bis die \textbf{Gauß-Normalform} erreicht ist
\[
    (\hat{A}, \hat{b}) = \begin{pmatrix}
        1 & \dots & 0 & \mid & \hat{a}_{r,r+1} & \dots & \hat{a}_{1n} &\mid & \hat{b}_1\\
        & \ddots & & \mid & & \vdots & & \mid\\
        0 & \dots & 1 & \mid & \hat{a}_{r,r+1} & \dots &  \hat{a}_{rn} & \mid & \hat{b}_r\\
        \hline\\
        \multicolumn{9}{c}{\vdots}\\
        \multicolumn{9}{c}{0}\\
        \multicolumn{9}{c}{\vdots}
    \end{pmatrix} 
.\]
Seien nun folgende Vektoren definiert
\[
    \hat{b}:=\begin{pmatrix}
        \hat{b}_1\\
        \vdots\\
        \hat{b}_r\\
        0\\
        \vdots\\
        0
    \end{pmatrix},
    \mu_1 := \begin{pmatrix}
        -\hat{a}_{1,r+1}\\
        \vdots\\
        -\hat{a}_{r,r+1}\\
        1\\
        0\\
        \vdots\\
        0
    \end{pmatrix},\dots,
    \mu_{n-r} := \begin{pmatrix}
        -\hat{a}_{1n}\\
        \vdots\\
        -\hat{a}_{rm}\\
        0\\
        \vdots\\
        0\\
        1
    \end{pmatrix}
.\]
\begin{theorem}[Lösung des Gleichungssystems]
    Sei $Ax = b$ ein lineares Gleichungssystem mit $A\in M(n\times n), b\in \mathbb{R}^m$.
    Dann gilt
    \[
        \text{Lös}(A,b)=\{ x \in \mathbb{R}^m\ \mid\ x = \hat{b} + \lambda_1\mu_1 + \dots \lambda_{n-r}\mu_{n-r}\ \text{mit}\ \lambda_1,\dots,\lambda_{r-n} \in \mathbb{R} \}    
    .\]
\end{theorem}
    \pagebreak
    \section{Frage 6}
\textbf{TODO: Meint er nur ein Beispiel oder Lösung mittels Gauß?}\\
\textit{Beispiele zu linearen Gleichungssystemen.}
\begin{example}
    Lineares Gleichungssystem als Gleichungen.
    \begin{align*}
        2x_1 + x_2 - 2x_3 + 3x_4 &= 4\\
        3x_1 + 2x_2 - x_3 + 2x_4 &= 6\\
        3x_1 + 3x_2 + 3x_3- 3x_4 &= 6
    \end{align*}
\end{example}
    \pagebreak
    \section{Frage 7}
\textit{Formulieren Sie genau die Definition einer Basis.}

\begin{definition}[Basis]
    Sei $V$ ein Vektorraum. Das $n$-Tupel ($x_1, \dots, x_n$) von Vektoren aus $V$ heißt
    (geordnete) Basis von $V$, wenn gilt:
    \begin{enumerate}
        \item $x_1,\dots,x_n$ sind linear unabhängig
        \item $\mathrm{lin}\{x_1,\dots,x_n\} = V$ (lineare Hülle)
    \end{enumerate}
\end{definition}
\begin{definition}[linear abhängig]
    Die Vektoren $x_1,\dots,x_n$ heißen \underline{linear} \underline{abhängig}, wenn es Skalare
    $\lambda_1,\dots,\lambda_n\in \mathbb{K}$ gibt, die \textbf{nicht alle} $0$ sind, sodass
    \[
        0 = \lambda_1 x_1 + \dots + \lambda_n x_n.
    \]
    Also: $0$ lässt sich als nicht-triviale Linearkombination von $x_1,\dots,x_n$ dar\-stellen.

    Die Vektoren heißen außerdem \underline{linear unabhängig}, wenn sie nicht linear abhängig
    sind.
\end{definition}
\begin{definition}[lineare Hülle]
    Die Menge aller Linearkombinationen von $x_1,\dots,x_n$
    \[
        \mathrm{lin}\{x_1,\dots,x_n\} := \{
            x = \lambda_1 x_1 + \dots + \lambda_n x_n\ \mid\ \lambda_1,\dots,\lambda_n\in \mathbb{K}
        \}
    \]
    heißt die lineare Hülle von $x_1,\dots,x_n$.
\end{definition}
    \pagebreak
    \section{Frage 8}
\textit{Formulieren Sie die wichtigsten Sätze zur Basis.}

\begin{theorem}
    ($x_1,\dots,x_n$) ist eine Basis von $V$ $\Longleftrightarrow$ 
    $\forall x \in V$ gibt es \underline{eindeutig} bestimmte 
    $\lambda_1,\dots,\lambda_n\in\mathbb{K}$ mit 
    \[
        x = \lambda_1 x_1 + \dots + \lambda_n x_n    
    .\]
\end{theorem}
\begin{theorem}[Invarianz der Basislänge]
    Besitzt ein Vektorraum $V$ eine Basis, dann haben alle Basen von $V$
    gleichviele Elemente.
\end{theorem}
\begin{theorem}[Basisauswahlsatz]
    Sei $V$ endlich erzeugt, also $V=\mathrm{lin}\{v_1,\dots,v_n\}$.
    Dann erhält man durch Weglassen geeigneter Elemente von 
    $\{v_1,\dots,v_n\}$ eine Basis von $V$.
\end{theorem}
\begin{theorem}[Basisergänzungssatz]
    Sei $V$ endlich erzeugt und seien $v_1,\dots,v_k$ linear unabhängige
    Vektoren von $V$.\\
    Dann lassen sich $v_1,\dots,v_k$ zu einer Basis ergänzen, d.h. es gibt\\ 
    $v_{k+1},\dots,v_{k+e}\ (e\geq 0)$, sodass $v_1,\dots,v_k,v_{k+1},\dots,v_{k+e}$
    eine Basis von $V$ bildet.
\end{theorem}
\begin{corollary}
    Jeder endlich erzeugte Vektorraum $V$ besitzt eine Basis.
\end{corollary}
\begin{corollary}
    Sei $V$ ein Vektorraum und $\mathrm{dim}\ V=k$. Seien weiters $v_1,\dots,v_k\in V$.
    Dann sind folgende Aussagen \underline{äquivalent}:
    \begin{enumerate}
        \item ($v_1,\dots,v_k$) ist eine Basis von $V$,
        \item $\mathrm{lin}\{v_1,\dots,v_k\} = V$,
        \item $v_1,\dots,v_k$ sind linear unabhängig.
    \end{enumerate}
\end{corollary}
\begin{corollary}
    Sei $V$ ein Vektorraum mit $\mathrm{dim}\ V=k$. Sei weiters $U$ ein Teilraum
    von $V$. Dann sind folgende Aussagen äquivalent:
    \begin{enumerate}
        \item $U=V$
        \item $\mathrm{dim}\ U=\mathrm{dim}\ V$
    \end{enumerate}
\end{corollary}
    \pagebreak
    \section{Frage 9}
\textit{Die Dimensionsformel für lineare Abbildungen: Formulierung und Beweis.}
\begin{theorem}[Dimensionsformel für lineare Abbildungen]
    Seien $V,W$ Vektorräume und $V$ endlich-dimensional. Sei $T:V\to W$
    eine lineare Abbildung. Dann folgt
    \[
        \mathrm{dim} (\mathrm{ker}\ T) + \mathrm{dim}(\mathrm{im}\ T) = \mathrm{dim}\ V
    .\]
\end{theorem}
\begin{proof}
    Sei $V$ endlich-dimensional mit $\mathrm{dim}\ V=n$, dann ist $\mathrm{ker}\ T$ ein 
    Teilraum von $V$, wodurch auch $\mathrm{ker}\ T$ endlich-dimensional ist. Sei
    weiters $v_1,\dots,v_k$ eine Basis von $\mathrm{ker}\ T$.
    Nach dem Basisergänzungssatz kann diese zu einer Basis $v_1,\dots,v_k,v_{k+1},\dots,v_n$
    von $V$ ergänzt werden.\\
    \textit{Behauptung}: $T(v_{k+1}),\dots,T(v_n)$ ist eine Basis von $\mathrm{im}\ T$.\\
    \begin{enumerate}
        \item[(1)] $\mathrm{lin}\{T(v_{k+1},\dots,T(v_{n}))\} = \mathrm{im}\ T$\\
        Sei $x\in V$. Da $v_1,\dots,v_n$ eine Basis von $V$ ist, gibt es 
        eindeutig bestimmte $\lambda_1,\dots,\lambda_n\in \mathbb{K}$, sodass
        \[
            x = \lambda_1v_1 + \dots + \lambda_n v_n    
        \]
        \begin{align*}
            \implies T(x) &= T(\lambda_1v_1 + \dots + \lambda_n v_n)\\
            &= \lambda_1 \underbrace{T(v_1)}_{=0} + \dots 
            + \lambda_k \underbrace{T(v_k)}_{=0} +
            \lambda_{k+1} T(v_{k+1}) + \dots +
            \lambda_n T(v_n)\\
            &=\lambda_{k+1} T(v_{k+1}) + \dots + \lambda_n T(v_n) 
            \in \mathrm{lin}\{T(v_{k+1},\dots,T(v_n))\}\\
            \implies \mathrm{im}\ T &\subseteq \mathrm{lin}\{
                T(v_{k+1},\dots,T(v_n))
            \}\\
            &\text{Klarerweise ist auch } \mathrm{lin}\{T(v_{k+1},\dots,T(v_n))\}\subseteq \mathrm{im}\ T\\
            \implies \mathrm{im}\ T &= \mathrm{lin}\{T(v_{k+1},\dots,T(v_n))\}
        \end{align*}
        \item[(2)] $T(v_{k+1},\dots,T(v_n))$ linear unabhängig:\\
        Sei $\lambda_{k+1}T(v_{k+1}) + \dots + \lambda_n T(v_n) = 0$
        \begin{align*}
            \implies &T(\lambda_{k+1}v_{k+1}+\dots +\lambda_nv_n) = 0\\
            \implies &\lambda_{k+1}v_{k+1} + \dots + \lambda_n v_n \in \mathrm{ker}\ T
        \end{align*}
        Da $v_1,\dots,v_k$ eine Basis von $\mathrm{ker}\ $T ist, gibt es 
        $\lambda_1,\dots,\lambda_n\in \mathbb{K}$ mit\\
        $\lambda_{k+1}v_{k+1} + \dots + \lambda_nv_n = \lambda_1 v_1 + \dots + \lambda_k v_k$.
        \[
            \implies \lambda_1v_1 + \dots + \lambda_kv_k - \lambda_{k+1}v_{k+1} - \dots - \lambda_n v_n = 0    
        .\]
        Da $v_1,\dots,v_n$ linear unabhängig sind, folgt
        \[
            \lambda_1 = \dots = \lambda_k = \lambda_{k+1} = \dots = \lambda_n = 0   
        .\]
        Also speziell $\lambda_{k+1} = \dots = \lambda_n = 0$
        \[
            \implies T(v_{k+1}),\dots,T(v_n)\ \text{sind l.u.}
        \]
        \item[(3)] $T(v_{k+1}),\dots,T(v_n)$ ist eine Basis von $\mathrm{im}\ T$
        \[
            \implies \mathrm{dim}(\mathrm{im}\ T) = n-k = \mathrm{dim}\ V - \mathrm{dim}(\mathrm{ker}\ T)    
        .\]
    \end{enumerate} 
\end{proof}
    \pagebreak
    \section{Frage 10}
\textit{Schnitt Gerade - Hyperebene: Beispiele}
\end{document}