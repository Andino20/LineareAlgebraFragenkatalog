\section{Frage 7}
\textit{Formulieren Sie genau die Definition einer Basis.}

\begin{definition}[Basis]
    Sei $V$ ein Vektorraum. Das $n$-Tupel ($x_1, \dots, x_n$) von Vektoren aus $V$ heißt
    (geordnete) Basis von $V$, wenn gilt:
    \begin{enumerate}
        \item $x_1,\dots,x_n$ sind linear unabhängig
        \item $\mathrm{lin}\{x_1,\dots,x_n\} = V$ (lineare Hülle)
    \end{enumerate}
\end{definition}
\begin{definition}[linear abhängig]
    Die Vektoren $x_1,\dots,x_n$ heißen \underline{linear} \underline{abhängig}, wenn es Skalare
    $\lambda_1,\dots,\lambda_n\in \mathbb{K}$ gibt, die \textbf{nicht alle} $0$ sind, sodass
    \[
        0 = \lambda_1 x_1 + \dots + \lambda_n x_n.
    \]
    Also: $0$ lässt sich als nicht-triviale Linearkombination von $x_1,\dots,x_n$ dar\-stellen.

    Die Vektoren heißen außerdem \underline{linear unabhängig}, wenn sie nicht linear abhängig
    sind.
\end{definition}
\begin{definition}[lineare Hülle]
    Die Menge aller Linearkombinationen von $x_1,\dots,x_n$
    \[
        \mathrm{lin}\{x_1,\dots,x_n\} := \{
            x = \lambda_1 x_1 + \dots + \lambda_n x_n\ \mid\ \lambda_1,\dots,\lambda_n\in \mathbb{K}
        \}
    \]
    heißt die lineare Hülle von $x_1,\dots,x_n$.
\end{definition}