\section{Frage 10}
\textit{Schnitt Gerade - Hyperebene: Beispiele}

Sei $G$ eine Gerade in Parameterform 
\[ 
    x=p+\lambda n = \begin{pmatrix}
        p_1\\\vdots\\p_k
    \end{pmatrix} + \lambda \begin{pmatrix}
        n_1\\ \vdots \\n_k
    \end{pmatrix}
.\]
Sei $H$ eine Hyperebene als Lösungsmenge der Gleichung
\[
    ax=b
\]
mit $a=(a_1,\dots,a_n), b\in \mathbb{R}$. Dann ist der Schnitt
$G \cap H$:
\begin{align*}
    &a(p + \lambda n) = b\\
    \implies &\lambda a n = b - ap
.\end{align*}
Dabei ist $a_n = (a_1,\dots,a_k)\begin{pmatrix}
    n1\\ \vdots\\n_k
\end{pmatrix} = a_1n_1+\dots + a_kn_k$ und $ap = a_1p_1 +\dots + a_kp_k$.
(Fall 1): $a_n \neq 0 \implies \lambda = \frac{b-ap}{an}$ und
\[
    x = p + \frac{b-ap}{an} n   
.\]
ist der Schnittpunkt von $G$ mit $H$.\\
(Fall 2): $an=0$\\
(2a): $b-ap = 0$ 
\begin{align*}
    \Longleftrightarrow\ & b = ap \Leftrightarrow p \in H\\
    \implies & \lambda an = b-ap \text{ gilt für alle } \lambda\in\mathbb{R}\\
    \implies & G\cap H = G
\end{align*}
(2b): $b-ap\neq 0 \Leftrightarrow p\notin H \implies G\cap H = \emptyset$
\begin{example}
    \begin{align*}
        G\ &: x = \begin{pmatrix}
            1\\ 1\\ 2
        \end{pmatrix} + \lambda \begin{pmatrix}
            2\\ 0\\ -1
        \end{pmatrix} = p + \lambda n\\
        H\ &: x_1 + 4x_2 = 1, \text{ also } a = \begin{pmatrix}
            1 & 4 & 0
        \end{pmatrix}, b = 1
    \end{align*}
    \underline{$G\cap H$}: $an = 2 \neq 0$
    \begin{align*}
        \implies x &= \begin{pmatrix}
            1\\ 1\\ 2
        \end{pmatrix} + \frac{1 - \begin{pmatrix} 1&4&0 \end{pmatrix}
        \begin{pmatrix}1\\1\\2\end{pmatrix}}{2}\begin{pmatrix}
            2\\0\\-1
        \end{pmatrix} = \begin{pmatrix}
            1\\1\\2
        \end{pmatrix} + \frac{1 - 5}{2} \begin{pmatrix}
            2\\0\\-1
        \end{pmatrix}\\
        x &=\begin{pmatrix}
            1\\1\\2
        \end{pmatrix} - 2 \begin{pmatrix}
            2\\0\\-1
        \end{pmatrix} = \begin{pmatrix}
            -3\\1\\4
        \end{pmatrix}\text{ \dots Schnitpunkt}
    \end{align*}
\end{example}