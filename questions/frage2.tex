\section{Frage 2}
\textit{Formulieren und beweisen Sie die Beziehung zwischen Matrizen und linearen Abbildungen}
\begin{theorem}
    Jede $m\times n$-Matrix $A$ entspricht in natürlicher Weise folgender Abbildungen
    \begin{align*}
        f_A : \mathbb{R}^n&\to \mathbb{R}^m\\
        x &\mapsto Ax
    \end{align*}
\end{theorem}
\begin{proof}
    Sei $Ax\in\mathbb{R}^m$ nach der Definition der Matrix-Vektormultiplikation.
    \begin{align*}
        f_A(x+y) &= A(x+y) = Ax + Ay = f_A(x) + f_A(y)\\
        f_A(\lambda x) &= A(\lambda x) = \lambda(Ax) = \lambda f_A(x)
    \end{align*}
\end{proof}
\begin{theorem}
    Zu jeder linearen Abbildung $T:\mathbb{R}^n\to \mathbb{R}^m$ gibt es eine \underline{eindeutig} 
    bestimmte Matrix $A\in M(m\times n)$ mit
    \[
        T=f_A
    \]
\end{theorem}
\begin{proof}
    Eindeutigkeit\\
    Seien $A,B \in M(m\times n)$ mit $T=f_A=f_B$. Dann ist zu zeigen, dass $A=B$. Anders ausgedrückt
    \begin{align*}
        T(x) &= f_A(x) = f_B(x)&\forall x\in \mathbb{R}^n\\
        \implies T(x) &= Ax = Bx &\forall x\in \mathbb{R}^n
    \end{align*}
    Wir betrachten speziell die Einheitsvektoren
    \[
        e_1 = \begin{pmatrix}
            1\\0\\\vdots\\ 0
        \end{pmatrix}, e_2 = \begin{pmatrix}
            0\\1\\\vdots\\0
        \end{pmatrix},\dots, e_n = \begin{pmatrix}
            0\\\vdots\\0\\1
        \end{pmatrix}\in\mathbb{R}^n
    \]
    \begin{align*}
        \implies &Ae_i = Be_i &\forall i = 1,\dots,n\\
        &Ae_i = \begin{pmatrix}
            a_{1i}\\a_{2i}\\\vdots\\a_{mi}
        \end{pmatrix} = Be_i=\begin{pmatrix}
            b_{1i}\\b_{2i}\\\vdots\\b_{mi}
        \end{pmatrix} & \forall i = 1,\dots,n\\
        \implies &a_{ki} = b_{ki}&\forall k=1,\dots,m\quad\forall i = 1,\dots,n\\
        \implies & A= B
    \end{align*}
\end{proof}
\begin{proof}
    Existenz\\
    Sei $T:\mathbb{R}^n\to \mathbb{R}^m$ linear. Wir setzen die Einheitsvektoren in die Abbildung $T$ ein.
    \begin{align*}
        Te_1 &= T \begin{pmatrix}
            1\\0\\\vdots\\0
        \end{pmatrix}=\begin{pmatrix}
            a_{11}\\\vdots\\ a_{m1}
        \end{pmatrix}\in\mathbb{R}^m\\
        Te_2 &= T \begin{pmatrix}
            0\\1\\\vdots\\0
        \end{pmatrix}=\begin{pmatrix}
            a_{12}\\\vdots\\ a_{m2}
        \end{pmatrix}\in\mathbb{R}^m\\
        &\vdots\\
        Te_n &= T \begin{pmatrix}
            0\\\vdots\\0\\1
        \end{pmatrix}=\begin{pmatrix}
            a_{1n}\\\vdots\\ a_{mn}
        \end{pmatrix}\in\mathbb{R}^m
    \end{align*}
    Wir definieren nun die Matrix $A$ durch
    \[
        A := \begin{pmatrix}
            a_{11} & a_{12} & \dots & a_{1n}\\
            \vdots & \vdots& &\vdots\\
            a_{m1} & a_{m2} & \dots & a_{mn}
        \end{pmatrix} = \begin{pmatrix}
            Te_1 & Te_2 & \dots & Te_n
        \end{pmatrix}
    \]
    wobei $A\in M(m\times n)$. Jetzt müssen wir \underline{noch zeigen} $f_A = T$. Sei $x = \begin{pmatrix}
        x_1\\\vdots\\ x_n
    \end{pmatrix}\in\mathbb{R}^n$.
    \begin{align*}
        \implies x &= \begin{pmatrix}
            x_1\\\vdots\\ x_n
        \end{pmatrix} = \begin{pmatrix}
            x_1\\0\\\vdots\\0
        \end{pmatrix} + \dots + \begin{pmatrix}
            0\\\vdots\\0 \\x_n
        \end{pmatrix} = x_1 \begin{pmatrix}
            1\\0\\\vdots\\0
        \end{pmatrix} + \dots + x_n \begin{pmatrix}
            0\\\vdots\\0\\1
        \end{pmatrix}\\
        x &= x_1e_1 + \dots + x_ne_n\\
        \implies f_A(x) &= f_A(x_1e_1 + \dots + x_ne_n) \underset{f_A \text{ linear}}{=} x_1f_A(e_1) + \dots + x_nf_A(e_n)\\
        &= x_1Ae_1 + \dots + x_nAe_n = x_1 \begin{pmatrix}
            a_{11}\\\vdots\\a_{m1}
        \end{pmatrix} + \dots + x_n \begin{pmatrix}
            a_{1n}\\\vdots\\a_{mn}
        \end{pmatrix}\\
        &= x_1T(e_1) + \dots + x_nT(e_n)\quad(\text{Definition von } A)\\
        &= T(x_1e_1 + \dots + x_ne_n) = T(x)\quad (T \text{ ist linear})\\
        \biimplies f_A(x) &= T(x)\qquad\forall x \in \mathbb{R}^n
    \end{align*}
\end{proof}