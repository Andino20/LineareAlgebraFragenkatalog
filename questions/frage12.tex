\section{Frage 12}
\textit{Die inverse Matrix: Definition, Berechnung der inversen Matrix und Beispiele.}
\begin{definition}[invertierbare Matrix]
    Eine $n\times n$-Matrix heißt \underline{invertierbar}, wenn
    \begin{align*}
        f_A\ :\ &\mathbb{R}^n \to \mathbb{R}^n\\
        & f_A(x) = Ax
    \end{align*}
    bijektiv ist.
\end{definition}
\begin{corollary}
    $A\in M(n\times n)$ ist invertierbar genau dann, wenn $\text{rg}(A)=n$.
\end{corollary}
\begin{definition}[Inverse Matrix]
    Sei $A$ eine $n\times n$-Matrix, dann heißt $A^{-1}$ die dazu inverse Matrix, wenn
    \[
        A^{-1}\cdot A = E
    \]
    gilt.
\end{definition}
\subsection*{Berechnung der inversen Matrix}
Suche die Spaltenvektoren $a_1,\dots,a_n$ von $A^{-1} = \begin{pmatrix}
    a_1 & \dots & a_n
\end{pmatrix}$. Da $AA^{-1}= E$ folgt $\forall i=1,\dots,n$
\[
    Aa_i=e_i
.\]
Da $\text{rg}(A) = n$, ist $Ax = e_i$ eindeutig lösbar $\forall i = 1,\dots, n$. Weiters ist $a_i$ die eindeutige
Lösung von 
\[
    Ax = e_i\qquad\forall i=1,\dots n.
\]
Dadurch entstehen $n$ Gleichungssysteme, die simultan gelöst werden können, indem man die Vektoren $e_1, \dots, e_n$
nebeneinander hinter die Matrix $A$ schreibt und dann den Gauß-Algorithmus anwendet.
\begin{multline*}    
    \left(
        \begin{array}{c | c c c c}
            A & e_1 & e_2 & \dots & e_n
        \end{array}
    \right) = \left(\begin{array}{c | c c c}
        \multirow{4}{*}{\Huge$A$} & 1 & & 0\\
        & & \ddots &\\
        & 0 & & 1
    \end{array}\right) \xrightarrow[]{\text{Gauß}}
    \left(
        \begin{array}{c c c | c c c}
            1 & & 0 & \multirow{3}{*}{$a_1$} & \multirow{3}{*}{$\dots$} & \multirow{3}{*}{$a_n$}\\
              & \ddots &\\
            0 & & 1
        \end{array}
    \right)\\
    =\left(
        \begin{array}{ccc|r}
            1 & & 0 & \multirow{4}{*}{\Huge$A^{-1}$}\\
              & \ddots &\\
            0 & & 1 &
        \end{array}
    \right)
\end{multline*}
\begin{example}
    Sei $A$ eine $4\times 4$-Matrix mit
    \[
        A = \begin{pmatrix}
            1 & 0 & 1 & 1\\
            1 & 1 & 2 & 1\\
            0 & -1 & 0 & 1\\
            1 & 0 & 0 & 2
        \end{pmatrix}
    \]
    
    \begin{elimination}[4]{4}{0.8em}{1.1}
        \eliminationstep
        {
            1 & 0 & 1 & 1 & 1 & 0 & 0 & 0\\
            1 & 1 & 2 & 1 & 0 & 1 & 0 & 0\\
            0 & -1 & 0 & 1 & 0 & 0 & 1 & 0\\
            1 & 0 & 0 & 2 & 0 & 0 & 0 & 1
        }
        {
            \text{II} - \text{I} \to \text{II}\\
            \text{IV} - \text{I} \to \text{IV}
        }
        \eliminationstep
        {
            1 & 0 & 1 & 1 & 1 & 0 & 0 & 0\\
            0 & 1 & 1 & 0 & -1 & 1 & 0 & 0\\
            0 & -1 & 0 & 1 & 0 & 0 & 1 & 0\\
            0 & 0 & -1 & -1 & -1 & 0 & 0 & 1
        }
        {
            \text{II} + \text{III} \to \text{III}
        }
        \\[10pt]
        \eliminationstep
        {
            1 & 0 & 1 & 1 & 1 & 0 & 0 & 0\\
            0 & 1 & 1 & 0 & -1 & 1 & 0 & 0\\
            0 & 0 & 1 & 1 & -1 & 1 & 1 & 0\\
            0 & 0 & -1 & 1 & -1 & 0 & 0 & 1
        }
        {
            \text{III} + \text{IV} \to \text{IV}
        }
        \eliminationstep
        {
            1 & 0 & 1 & 1 & 1 & 0 & 0 & 0\\
            0 & 1 & 1 & 0 & -1 & 1 & 0 & 0\\
            0 & 0 & 1 & 1 & -1 & 1 & 1 & 0\\
            0 & 0 & 0 & 2 & -2 & 1 & 1 & 0
        }
        {
            \text{I} - \text{III} \to \text{I}\\
            \text{II} - \text{III} \to \text{II}\\
            \nicefrac{1}{2}\cdot \text{IV}
        }
        \\[10pt]
        \eliminationstep
        {
            1 & 0 & 0 & 0 & 2 & -1 & -1 & 0\\
            0 & 1 & 0 & -1 & 0 & 0 & -1 & 0\\
            0 & 0 & 1 & 1 & -1 & 1 & 1 & 0\\
            0 & 0 & 0 & 1 & -1 & \frac{1}{2} & \frac{1}{2} & \frac{1}{2}
        }
        {
            \text{II} + \text{IV} \to \text{II}\\
            \text{III} - \text{IV} \to \text{III}
        }
        \eliminationstep
        {
            1 & 0 & 0 & 0 & 2 & -1 & -1 & 0\\
            0 & 1 & 0 & 0 & -1 & \frac{1}{2} & -\frac{1}{2} & \frac{1}{2}\\
            0 & 0 & 1 & 0 & 0 & \frac{1}{2} & \frac{1}{2} & -\frac{1}{2}\\
            0 & 0 & 0 & 1 & -1 & \frac{1}{2} & \frac{1}{2} & \frac{1}{2}
        }
        {
            \\
        }
    \end{elimination}
    \[
        \implies A^{-1} = \begin{pmatrix}
            2 & -1 & 1 & 0\\
            -1 & \nicefrac{1}{2} & -\nicefrac{1}{2} & \nicefrac{1}{2}\\
            0 & \nicefrac{1}{2} & \nicefrac{1}{2} & -\nicefrac{1}{2}\\
            -1 & \nicefrac{1}{2} & \nicefrac{1}{2} & \nicefrac{1}{2}
        \end{pmatrix}
    \]
\end{example}