\section{Definitionen u. Ergänzungen}
\begin{definition}
    [Norm] Sei $V$ ein $\mathbb{K}$-Vektorraum. Eine Abbildung
    \[
        \norm{\ .\ }:V\to\mathbb{R}    
    \]
    heißt \textbf{Norm}, wenn folgendes gilt:
    \begin{enumerate}
        \item[(1)] $\forall x \in V :$
        \begin{gather*}
            \norm{x} \geq 0\\
            \norm{x} = 0 \Leftrightarrow x = 0
        \end{gather*}
        \item[(2)] $\forall x\in V,\forall \lambda \in \mathbb{K}:$
        \[
            \norm{\lambda x} = \vert \lambda \vert\norm{x}    
        \]
        \item[(3)] Dreiecksungleichung $\forall x,y\in V :$
        \[
            \norm{x + y} \leq \norm{x} + \norm{y}    
        \]
    \end{enumerate}
\end{definition}
\begin{definition}
    [euklidische Norm] Sei $V$ ein euklidischer Vektorraum, dann ist
    \[
        \norm{x}:=\sqrt{\inner{x}{x}}\qquad\forall x\in V    
    \]
    eine Norm. $\norm{x}$ heißt dann die euklidische Norm (oder $L^2$-Norm).
\end{definition}
\begin{definition}
    [Koordinatenvektor] Sei $\{v_1,\dots,v_n\}$ eine Orthonormalbasis. Dann ist der
    Koordinatenvektor $x_B$ von $x\in V$ bezüglich der Orthonormalbasis 
    $B=\{v_1,\dots,v_n\}$
    \[
        x_B = \begin{pmatrix}
            \inner{x}{v_1}\\
            \vdots\\
            \inner{x}{v_n}
        \end{pmatrix}
    .\] $\inner{x}{v_i}$ heißt auch der $i$-te Fourierkoeffizient von $x$.
\end{definition}
\begin{definition}
    [orthogonales Komplement] Sei $V$ ein euklidischer Vektorraum, $U$ ein Teilraum
    von $V$. Dann heißt
    \[
        U^\bot := \{x\in V\mid \inner{x}{u}=0\qquad\forall u \in U\}    
    \]
    das \underline{orthogonale Komplement} von $U$.
\end{definition}
\begin{definition}
    [orthogonale Projektion] Sei $V$ ein endlichdimensionaler euklidischer Vektorraum 
    und $U\subseteq V$ ein Teilraum, also:
    \[
        \forall x \in V\ \exists! x_1\in U, x_2\in U^\bot\text{ mit } x = x_1 + x_2
    \]
    Dann heißt 
    \begin{align*}
        \pi_U : V &\to U\\
        x&\mapsto x_2
    \end{align*}
    die \underline{orthogonale Projektion auf $U$} und
    \begin{align*}
        \pi_{U^\bot} : V &\to U^\bot\\
        x &\mapsto x_2
    \end{align*}
    die \underline{orthogonale Projektion auf $U^\bot$}.
\end{definition}
\begin{definition}
    [Kern und Bild v. linearen Abbildungen] Seien $V,W$ Vektorräume und $T:~V\to~W$ eine lineare Abbildung.
    \begin{align*}
        \mathrm{im}\ T &:=\{T(x)\mid x\in V\} = \{y\in W\mid \exists x \in V \text{ mit } T(x)=y\}\qquad\text{\dots Bild von }T\\
        \mathrm{ker}\ T &:= \{x\in V\mid T(x) = 0\}\qquad \dots\text{Kern von }T
    \end{align*}
    Weiters gilt:
    \begin{enumerate}
        \item[(a)] $\mathrm{im}\ T$ ist ein Teilraum von $W$.
        \item[(b)] $\mathrm{ker}\ T$ ist ein Teilraum von $V$.
    \end{enumerate}
\end{definition}
\begin{definition}
    [Winkel zwischen zwei Vektoren] Sei $\inner{\ .\ }{\ .\ } :V\to\mathbb{R}$ das reelle Standard-skalarprodukt
    des Vektorraums $V$ und $x,y\in V$. Dann gilt
    \[
        \inner{x}{y} = \cos \varphi\norm{x}\norm{y} = \cos \varphi \sqrt{\inner{x}{x}}\sqrt{\inner{y}{y}}.
    \]
    Durch Äquivalenzumformungen erhält man für den Winkel $\varphi$ zwischen den Vektoren $x,y$
    \[
        \varphi = \arccos \left(\frac{
            \inner{x}{y}
        }{
            \norm{x}\norm{y}
        }\right)
    \]
\end{definition}
\begin{definition}
    [geometrische Vielfachheit] Die Dimension eines Eigenraums $\mathrm{dim}\ E_\lambda$ heißt
    die \underline{geometrische Vielfachheit} von $\lambda$.
\end{definition}
\begin{definition}
    [Gruppe] Sei $G$ eine Menge. Sei weiters
    \begin{align*}
        +: G\times G &\to G\\
        (a,b)&\mapsto a + b
    \end{align*}
    eine Verknüpfung über der Menge $G$. Man nennt $(G,+)$ eine \underline{Gruppe}, wenn folgende Gesetze erfüllt sind
    \begin{enumerate}
        \item[(1)] Assoziativitätsgesetz.
        \[
            \forall a,b,c\in G : a + (b + c) = (a + b) + c
        \]
        \item[(2)] Neutrales Element
        \[
            \exists e \in G, \forall a\in G : a + e = a
        \]
        \item[(3)] Inverses Element 
        \[
            \forall a \in G, \exists a^{-1}\in G : a + a^{-1} = e
        \]
    \end{enumerate}
    Gilt außerdem noch das Kommutativgesetz
    \[
        \forall a,b\in G : a + b = b + a
    \]
    sagt man auch \underline{abelsche Gruppe} dazu.
\end{definition}
\begin{definition}
    [Körper] Ein Körper ist ein Tupel $(\mathbb{K}, +, \cdot)$, wobei $\mathbb{K}$ eine Menge und $+,\cdot$ zwei Verknüpfungen über
    der Grundmenge sind
    \begin{align*}
        + : \mathbb{K} \times \mathbb{K} &\to \mathbb{K}\\
        (a,b) &\mapsto a + b\\
        \cdot : \mathbb{K} \times \mathbb{K} &\to \mathbb{K}\\
        (a,b)&\mapsto a\cdot b
    \end{align*}
    und außerdem folgende Bedingungen gelten
    \begin{enumerate}
        \item $(K,+)$ ist eine abelsche Gruppe mit dem neutralen Element $0$,
        \item $(K\backslash\{0\}, \cdot)$ ist eine abelsche Gruppe mit dem neutralen Element $1$,
        \item Das Distributivitätsgesetz gilt:
        \[
            a \cdot (b + c) =  a\cdot b + a\cdot c\text{ und } (a + b) \cdot c = a\cdot c + b\cdot c\qquad\forall a,b,c\in\mathbb{K}
        \]
    \end{enumerate}
\end{definition}
\begin{definition}
    [Matrix] Eine $m\times n$-Matrix $A$ ist eine Anordnung von $m\cdot n$ Elementen nach folgendem Rechteckschema
    \[
        A = \begin{pmatrix}
            a_{11} & a_{12} & \dots & a_{1n}\\
            a_{21} & a_{22} & \dots & a_{2n}\\
            \vdots & & & \vdots\\
            a_{m1} & a_{m2} & \dots & a_{mn}
        \end{pmatrix}.
    \]
    Schreibweise: $A=(a_{ij})^{m,n}_{i,j=1} = (a_{ij})$.\\
    $a_{ij}$ heißt der \underline{Koeffizient} der Matrix A in der $i$-ten Zeile und der $j$-ten Spalte.
    \begin{align*}
        \begin{pmatrix}
            a_{i1} & \dots & a_{1n}
        \end{pmatrix}\qquad&(i = 1,\dots,m)\quad\text{Zeilen}\\
        \begin{pmatrix}
            a_{1j}\\
            \vdots\\
            a_{mj}
        \end{pmatrix}\qquad&(j=1,\dots,n)\quad\text{Spalten}
    \end{align*}
    $M(m\times n)$ ist die Menge aller $m\times n$-Matrizen.
\end{definition}
\begin{definition}
    [Komplexität der Matrixmultiplikation] Bei zwei quadratischen Matrizen ist die Laufzeit 
    der Multiplikationen $\mathcal{O}(n^3)$.
\end{definition}