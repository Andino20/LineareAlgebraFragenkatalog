\section{Definitionen}
\begin{definition}
    [Norm] Sei $V$ ein $\mathbb{K}$-Vektorraum. Eine Abbildung
    \[
        \norm{\ .\ }:V\to\mathbb{R}    
    \]
    heißt \textbf{Norm}, wenn folgendes gilt:
    \begin{enumerate}
        \item[(1)] $\forall x \in V :$
        \begin{gather*}
            \norm{x} \geq 0\\
            \norm{x} = 0 \Leftrightarrow x = 0
        \end{gather*}
        \item[(2)] $\forall x\in V,\forall \lambda \in \mathbb{K}:$
        \[
            \norm{\lambda x} = \vert \lambda \vert\norm{x}    
        \]
        \item[(3)] Dreiecksungleichung $\forall x,y\in V :$
        \[
            \norm{x + y} \leq \norm{x} + \norm{y}    
        \]
    \end{enumerate}
\end{definition}
\begin{definition}
    [euklidische Norm] Sei $V$ ein euklidischer Vektorraum, dann ist
    \[
        \norm{x}:=\sqrt{\inner{x}{x}}\qquad\forall x\in V    
    \]
    eine Norm. $\norm{x}$ heißt dann die euklidische Norm (oder $L^2$-Norm).
\end{definition}
\begin{definition}
    [Koordinatenvektor] Sei $\{v_1,\dots,v_n\}$ eine Orthonormalbasis. Dann ist der
    Koordinatenvektor $x_B$ von $x\in V$ bezüglich der Orthonormalbasis 
    $B=\{v_1,\dots,v_n\}$
    \[
        x_B = \begin{pmatrix}
            \inner{x}{v_1}\\
            \vdots\\
            \inner{x}{v_n}
        \end{pmatrix}
    .\] $\inner{x}{v_i}$ heißt auch der $i$-te Fourierkoeffizient von $x$.
\end{definition}
\begin{definition}
    [orthogonales Komplement] Sei $V$ ein euklidischer Vektorraum, $U$ ein Teilraum
    von $V$. Dann heißt
    \[
        U^\bot := \{x\in V\mid \inner{x}{u}=0\qquad\forall u \in U\}    
    \]
    das \underline{orthogonale Komplement} von $U$.
\end{definition}