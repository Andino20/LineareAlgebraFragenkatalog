\section{Frage 11}
\textit{Schnitt von zwei affinen Teilräumen: Beispiele}
\begin{gather*}
    L\ :\ x =\begin{pmatrix}
        2\\1\\3\\0
    \end{pmatrix} + \lambda_1\begin{pmatrix}
        1\\2\\1\\0
    \end{pmatrix} + \lambda_2 \begin{pmatrix}
        0\\1\\1\\2
    \end{pmatrix}\qquad\text{Ebene im }\mathbb{R}^4\\
    H\ :\ x_1 + x_2 + x_3 + x_4 = 5\qquad\text{Hyperebene}
\end{gather*}
\underline{Schritt 1}: Bestimme die transponierte Matrix $C^T$ aus alle Vektoren von $L$ (außer dem 1.) und bringe sie auf die Gauß-Normalform.
\[
    C = \begin{pmatrix}
        1 & 0\\
        2 & 1\\
        1 & 1\\
        0 & 2
    \end{pmatrix} \implies C^T = \begin{pmatrix}
        1 & 2 & 1 & 0\\
        0 & 1 & 1 & 2
    \end{pmatrix}
\]
\begin{elimination}{4}{1.1em}{1.1}
    \eliminationstep
    {
        1 & 2 & 1 & 0\\
        0 & 1 & 1 & 2
    }
    {
        -(2)\mathrm{II} + \mathrm{I} \to \mathrm{I}
    }
    \eliminationstep
    {
        1 & 0 & -1 & -4\\
        0 & 1 & 1 & 2
    }
    {
        \\
    }
\end{elimination}
\underline{Schritt 2}: Aus der Gauß-Normalform die Lösungsvektoren $a_i^T$ auslesen und die transponierten Vektoren
in die Matrix $A$ schreiben.
\begin{gather*}
    a_1^T=\begin{pmatrix}
        1\\-1\\1\\0
    \end{pmatrix}, a_2^T =\begin{pmatrix}
        4\\-2\\0\\1
    \end{pmatrix}\\
    \implies A = \begin{pmatrix}
        a_1\\a_2
    \end{pmatrix}=\begin{pmatrix}
        1 & -1 & 1 & 0\\
        4 & -2 & 0 & 1
    \end{pmatrix}
\end{gather*}
\underline{Schritt 3}: Den inhomogenen Vektor $b$ berechnen, indem man $A$ mit dem 1. Vektor $p$ aus $L$ multipliziert.
\[
    b = Ap = \begin{pmatrix}
        1 & -1 & 1 & 0\\
        4 & -2 & 0 & 1
    \end{pmatrix}\begin{pmatrix}
        2\\ 1\\3\\0
    \end{pmatrix} = \begin{pmatrix}
        4\\6
    \end{pmatrix}
\]
\underline{Schritt 4}: Das Gleichungssystem für $L$ aufstellen und die Gleichung der Hyperebene $H$ anhängen.
\begin{gather*}
    \sysdelim..
    \systeme{
        x_1 - x_2 + x_3 = 4,
        4x_1 - 2x_2 + x_4 = 6,
        x_1 + x_2 + x_3 + x_4 = 5
    }
\end{gather*}
\underline{Schritt 5}: Das Gleichungssystem mittel Gauß-Elimination lösen.
\begin{elimination}[6]{4}{1.1em}{1.1}
    \eliminationstep
    {
        1 & -1 & 1 & 0 & 4\\
        4 & -2 & 0 & 1 & 6\\
        1 & 1 & 1 & 1 & 5
    }
    {
        (-4)\text{I} + \text{II} \to \text{II}\\
        (-1)\text{I} + \text{III} \to \text{III}
    }
    \eliminationstep
    {
        1 & -1 & 1 & 0 & 4\\
        0 & 2 & -4 & 1 & -10\\
        0 & 2 & 0 & 1 & 1
    }
    {
        (-1)\text{II} + \text{III} \to \text{III}\\
        \nicefrac{1}{2}\cdot\text{III}
    }
    \\[10pt]
    \eliminationstep
    {
        1 & -1 & 1 & 0 & 4\\
        0 & 1 & -2 & \nicefrac{1}{2} & -5\\
        0 & 0 & 4 & 0 & 11
    }
    {
        \nicefrac{1}{4}\cdot \text{III}
    }
    \eliminationstep
    {
        1 & -1 & 1 & 0 & 4\\
        0 & 1 & -2 & \nicefrac{1}{2} & -5\\
        0 & 0 & 1 & 0 & \nicefrac{11}{4}
    }
    {
        \text{I} + \text{II} \to \text{I}
    }
    \\[10pt]
    \eliminationstep
    {
        1 & 0 & -1 & \nicefrac{1}{2} & -1\\
        0 & 1 & -2 & \nicefrac{1}{2} & -5\\
        0 & 0 & 1 & 0 & \nicefrac{11}{4}
    }
    {
        \text{I} + \text{III} \to \text{I}\\
        \text{II} + 2\cdot\text{III} \to \text{II}
    }
    \eliminationstep
    {
        1 & 0 & 0 & \nicefrac{1}{2} & \nicefrac{7}{4}\\
        0 & 1 & 0 & \nicefrac{1}{2} & \nicefrac{1}{2}\\
        0 & 0 & 1 & 0 & \nicefrac{11}{4}
    }
    {
        \\
    }
\end{elimination}
\[
    \implies x = \begin{pmatrix}
        \nicefrac{7}{4}\\
        \nicefrac{1}{2}\\
        \nicefrac{11}{4}\\
        0
    \end{pmatrix} + \lambda \begin{pmatrix}
        -\nicefrac{1}{2}\\
        -\nicefrac{1}{2}\\
        0\\
        1
    \end{pmatrix}
\]
Daraus folgt, dass $L\cap H$ eine Gerade durch den $\mathbb{R}^4$ ist.