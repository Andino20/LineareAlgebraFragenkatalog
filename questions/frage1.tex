\section{Frage 1}
\textit{Vektorraum: Genaue Definition und Beispiele von Vektorräumen.}
\begin{definition}
    Vektorraum\\
    Sei $V$ eine nichtleere Menge mit zwei Abbildungen
    \begin{align*}
        +: &\ V \times V \to V &&\text{(Addition)}\\
        & (x,y) \to x+y\\
        \cdot :\ & \mathbb{K} \times V \to V &&\text{(Skalarmultiplikation)}\\
        & (\lambda, x)\to \lambda x
    \end{align*}
    dann heißt $(V, +, \cdot)$ ein $\mathbb{K}$-Vektorraumm, wenn folgende Axiome gelten:
    \begin{enumerate}
        \item[(V1)] Abgeschlossenheit bzg. +
        \[
            x,y\in V \implies x + y\in V
        \]
        \item[(V2)] Assoziativität
        \[
            x + (y + z) = (x + y) + z\qquad \forall x,y,z\in V
        \]
        \item[(V3)] Neutrales Element\\
        Es gibt ein Element $0\in V$ mit
        \[
            x + 0 = x\qquad \forall x\in V
        \]
        \item[(V4)] Inverses Element\\
        Zu jedem $x\in V$ gibt es ein Element $(-x)\in V$ mit
        \[
            x + (-x) = 0
        \]
        \item[(V5)] Kommutativität
        \[
            x+y = y+x\qquad \forall x,y\in V
        \]
        \item[(V6)] Abgeschlossenheit bzg. $\cdot$
        \[
            x\in V, \lambda \in \mathbb{K}\implies \lambda x \in V
        \]
        \item[(V7)] $\forall x,y\in V, \forall \lambda \in \mathbb{K}$
        \[
            \lambda(x+y)= \lambda x + \lambda y
        \]
        \item[(V8)] $\forall x \in V, \forall \lambda, \mu \in \mathbb{K}$
        \[
            (\lambda + \mu)x = \lambda x + \mu x
        \]
        \item[(V9)] $\forall x \in V, \forall \lambda,\mu\in\mathbb{K}$
        \[
            \lambda(\mu\cdot x) = (\lambda\mu)\cdot x
        \]
        \item[(V10)] $\forall x\in V$
        \[
            1 \cdot x = x
        \]
    \end{enumerate}
\end{definition}
\begin{example}
    Vektorräume
    \begin{enumerate}
        \item Der $\mathbb{R}^n$
        \[
            \mathbb{R}^n=
            \left\{
            x =
            \begin{pmatrix}
                x_1 \\ \vdots \\ x_n
            \end{pmatrix} :
            x_i\in\mathbb{R}, \forall i = 1, \dots, n
            \right\}
        \]
        \item Sei $A$ eine nichtleere Menge.
        \[
            V := \{ f\ \mid\ f: A\to\mathbb{K} \text{ ist eine Funktion}\}
        \]
        Wir definieren $f+g$ und $\lambda f$ punktweise
        \begin{gather*}
            (f + g) (x) := f(x) + g(x)\\
            (\lambda f)(x) := \lambda \cdot f(x)
        \end{gather*}
    \end{enumerate}
\end{example}