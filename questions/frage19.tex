\section{Frage 19}
\textit{Eigenwerte: Definition von Eigenwert, Eigenvektor und Eigenraum.}
\begin{definition}
    [Eigenwert, -vektor] Sei $T:V\to V$ linear. Eine Zahl $\lambda \in K$ heißt \underline{Eigenwert} von $T$,
    wenn es einen Vektor $x\neq 0$ gibt, sodass
    \[
        T(x) = \lambda x.
    \]
    $x$ heißt dann \underline{Eigenvektor} von $T$ bezüglich $\lambda$.
\end{definition}
\begin{definition}
    [Eigenraum] Sei 
    \[
        E_\lambda := \{x\in V : x \text{ ist Eigenvektor zu }\lambda\}.
    \]
    Dann heißt $E_\lambda$ \underline{Eigenraum} zum Eigenwert $\lambda$.
\end{definition}