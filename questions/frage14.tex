\section{Frage 14}
\textit{Die Dreiecksungleichung für die Norm in einem euklidischen Vektorraum: 
Formulierung und Beweis.}

\begin{definition}[Dreiecksungleichung der Norm]
    Sei $\norm{\ .\ } : V \to \mathbb{R}$ eine Norm, dann gilt per dessen Definition
    \[
        \norm{x + y} \leq \norm{x} + \norm{y}\qquad\forall x,y\in V
    .\]
\end{definition}
\begin{proof} Dreiecksungleichung der Norm\\
    Sei $z = a+ib=\textrm{Re}(z) + \textrm{Im}(z)\in \mathbb{C} \implies \vert\mathrm{Re}(z)\vert \leq \vert z\vert$
    \begin{align*}
        \implies \norm{x+y}^2 &= \inner{x+y}{x+y}\\
        &= \inner{x}{x} + \inner{x}{y} + \inner{y}{x} + \inner{y}{y}\\
        &= \norm{x}^2 + \underbrace{\inner{x}{y} + \overline{\inner{x}{y}}}_{\text{Re}(z) = (z + \overline{z}) / 2} + \norm{y}^2\\
        &= \norm{x}^2 + 2\textrm{Re}(\inner{x}{y}) + \norm{y}^2\\
        &\leq \norm{x}^2 + 2\vert \textrm{Re}(\inner{x}{y}) \vert + \norm{y}^2\\
        &\leq \norm{x}^2 + 2\vert \inner{x}{y}\vert + \norm{y}^2\\
        \text{Cauchy-Schwarz} &\leq \norm{x}^2 + 2 \sqrt{\inner{x}{x}}\sqrt{\inner{y}{y}} + \norm{y}^2\\
        &=\norm{x}^2 + 2 \norm{x}\norm{y} + \norm{y}^2\\
        &= (\norm{x} + \norm{y})^2\\
        \implies \norm{x + y} &\leq \norm{x} + \norm{y}\qquad\forall x,y\in V
    \end{align*}
\end{proof}