\section{Frage 17}
\textit{Abstand Punkt - affiner Teilraum: Beispiele. (Siehe Formeln Dokumentende)}
\begin{example}
    Sei $G$ eine Gerade im $\mathbb{R}^3, G= p + U$ und sei $x$ ein Punkt im $\mathbb{R}^3$.
    \begin{align*}
        G &: x = p + \lambda u = \begin{pmatrix}
            1\\1\\1
        \end{pmatrix} + \lambda \begin{pmatrix}
            2\\0\\-1
        \end{pmatrix}\\
        x &:= \begin{pmatrix}
            1\\2\\3
        \end{pmatrix}
    \end{align*}
    Wir bestimmen den Basisvektor $v$ von $G$ und berechnen die Distanz $d(x,G)$.
    \begin{align*}
        v &= \frac{1}{\norm{u}}u = \frac{1}{\sqrt{5}}\begin{pmatrix}
            2\\0\\-1
        \end{pmatrix} \implies U = \mathrm{lin}\{v\}\\
        \implies d(x,G) &= \norm{x-p-\pi_U(x-p)}\\
        &= \norm{x -p - \inner{x-p}{v}v}\\
        &= \norm{\begin{pmatrix}
            0\\1\\2
        \end{pmatrix}+\frac{2}{5}\begin{pmatrix}
            2\\0\\-1
        \end{pmatrix}} = \frac{1}{5}\norm{\begin{pmatrix}
            0\\5\\10
        \end{pmatrix} - \begin{pmatrix}
            4\\0\\-2
        \end{pmatrix}} = \frac{1}{5}\begin{pmatrix}
            -4\\5\\12
        \end{pmatrix}\\
        &= \frac{1}{5}\sqrt{16 + 25 + 144} = \underline{\underline{\frac{1}{5}\sqrt{185}}}
    \end{align*}
\end{example}
\begin{example}
    Sei $H$ eine Hyperebene im $\mathbb{R}^3$ gegeben durch
    \[
        H:x_1 + 2x_2 - x_3 = 4 \implies a = \begin{pmatrix}
            1\\2\\-1
        \end{pmatrix}
    \]
    Sei $x = \begin{pmatrix}
        1\\2\\3
    \end{pmatrix}$.
    \begin{align*}
        d(x,H) &= \frac{\left\vert \inner{\begin{pmatrix}
            1\\2\\-1
        \end{pmatrix}}{\begin{pmatrix}
            1\\2\\3
        \end{pmatrix}} -4\right\vert}{\sqrt{6}}\\
        &= \frac{\vert 2 - 4\vert}{\sqrt{6}} = \frac{2}{\sqrt{6}} = \frac{2\sqrt{6}}{6} = \frac{\sqrt{6}}{3}\\
        d(x,H) &= \frac{\sqrt{6}}{3}
    \end{align*}
\end{example}