\section{Frage 8}
\textit{Formulieren Sie die wichtigsten Sätze zur Basis.}

\begin{theorem}
    ($x_1,\dots,x_n$) ist eine Basis von $V$ $\Longleftrightarrow$ 
    $\forall x \in V$ gibt es \underline{eindeutig} bestimmte 
    $\lambda_1,\dots,\lambda_n\in\mathbb{K}$ mit 
    \[
        x = \lambda_1 x_1 + \dots + \lambda_n x_n    
    .\]
\end{theorem}
\begin{theorem}[Invarianz der Basislänge]
    Besitzt ein Vektorraum $V$ eine Basis, dann haben alle Basen von $V$
    gleichviele Elemente.
\end{theorem}
\begin{theorem}[Basisauswahlsatz]
    Sei $V$ endlich erzeugt, also $V=\mathrm{lin}\{v_1,\dots,v_n\}$.
    Dann erhält man durch Weglassen geeigneter Elemente von 
    $\{v_1,\dots,v_n\}$ eine Basis von $V$.
\end{theorem}
\begin{theorem}[Basisergänzungssatz]
    Sei $V$ endlich erzeugt und seien $v_1,\dots,v_k$ linear unabhängige
    Vektoren von $V$.\\
    Dann lassen sich $v_1,\dots,v_k$ zu einer Basis ergänzen, d.h. es gibt\\ 
    $v_{k+1},\dots,v_{k+e}\ (e\geq 0)$, sodass $v_1,\dots,v_k,v_{k+1},\dots,v_{k+e}$
    eine Basis von $V$ bildet.
\end{theorem}
\begin{corollary}
    Jeder endlich erzeugte Vektorraum $V$ besitzt eine Basis.
\end{corollary}
\begin{corollary}
    Sei $V$ ein Vektorraum und $\mathrm{dim}\ V=k$. Seien weiters $v_1,\dots,v_k\in V$.
    Dann sind folgende Aussagen \underline{äquivalent}:
    \begin{enumerate}
        \item ($v_1,\dots,v_k$) ist eine Basis von $V$,
        \item $\mathrm{lin}\{v_1,\dots,v_k\} = V$,
        \item $v_1,\dots,v_k$ sind linear unabhängig.
    \end{enumerate}
\end{corollary}
\begin{corollary}
    Sei $V$ ein Vektorraum mit $\mathrm{dim}\ V=k$. Sei weiters $U$ ein Teilraum
    von $V$. Dann sind folgende Aussagen äquivalent:
    \begin{enumerate}
        \item $U=V$
        \item $\mathrm{dim}\ U=\mathrm{dim}\ V$
    \end{enumerate}
\end{corollary}