\section{Frage 15}
\textit{Orthonormalbasis: Genaue Definition: Beschreiben Sie die äquivalenten 
Formulierung einer Orthonormalbasis.}
\begin{definition}
    Sei $V$ ein euklidischer Vektorraum und seien $v_1,\dots,v_n\in V$.
    \begin{enumerate}
        \item[(1)] $\{v_1,\dots,v_n\}$ heißt \underline{Orthogonalsystem (OGS)}, wenn
        \[\inner{v_i}{v_j} = 0\qquad\forall i,j=1,\dots,n, i\neq j\] 
        \item[(2)] $\{v_1,\dots,v_n\}$ heißt \underline{Orthonormalsystem (ONS)}, wenn
        \[
            \inner{v_i}{v_j} = \begin{cases}
                1 &i = j\\
                0 &i \neq j
            \end{cases}\qquad\forall i,j =1,\dots,n
        .\]
        Also: $\{v_1,\dots,v_n\}$ ist OGS mit $\norm{v_i}=1, \forall i = 1,\dots,n$.
        \item[(3)] $\{v_1,\dots,v_n\}$ heißt \underline{Orthonormalbasis (ONB)}, wenn
        \begin{enumerate}
            \item $\{v_1,\dots,v_n\}$ ist ein Orthonormalsystem
            \item $\{v_1,\dots,v_n\}$ ist eine Basis von $V$
        \end{enumerate}
    \end{enumerate}
\end{definition}
\begin{corollary}
    $\{v_1,\dots,v_n\}$ ist eine Orthonormalbasis genau dann, wenn $\{v_1,\dots,v_n\}$
    ein Orthonormalsystem ist und $\mathrm{lin}\{v_1,\dots,v_n\}=V$.
\end{corollary}
\begin{theorem}[Fourier Entwicklung]
    Sei $V$ ein euklidischer Vektorraum und $\{v_1,\dots,v_n\}$ ein Orthonormalsystem 
    von $V$. Dann sind folgende Aussagen äquivalent:
    \begin{enumerate}
        \item[(1)] $\{v_1,\dots,v_n\}$ ist eine Orthonormalbasis von $V$
        \item[(2)] $\forall x\in V$ gilt:
        \[
            x = \sum_{i = 1}^{n}\inner{x}{v_i}v_i\qquad\text{Fourierdarstellung}  
        \]
        \item[(3)] $\forall x,y\in V$ gilt:
        \[
            \inner{x}{y} = \sum_{i=1}^{n}\inner{x}{v_i}\overline{\inner{y}{v_i}}
            \qquad\text{Parseval'sche Gleichung}   
        .\]
        \item[(4)] $\forall x \in V$ gilt:
        \[
            \norm{x}^2 = \sum_{i = 1}^{n}\vert \inner{x}{v_i}\vert^2
            \qquad \text{Parseval'sche Gleichung}    
        .\]
        \item[(5)] Ist $x\in V$ mit $\inner{x}{v_i} = 0$ für alle $i=1,\dots,n$, so gilt: $x=0$. 
    \end{enumerate}
\end{theorem}