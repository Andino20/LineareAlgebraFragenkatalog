\section{Frage 21}
\textit{Selbstadjungierte lineare Abbildungen und symmetrische Matrizen:
Definition. Formulieren Sie die Beziehung zwischen diesen beiden Begriffen.}
\begin{definition}
    [Selbstadjungierte Abbildung] Sei $V$ ein reeller euklidischer Vektorraum. Eine lineare Abbildung
    \[
        T: V\to V
    \] heißt \underline{selbstadjungiert}, wenn
    \[
    \inner{Tx}{y}=\inner{x}{Ty}\qquad \forall x,y\in V\]
\end{definition}
\begin{definition}
    [symmetrische Matrix] Eine reelle $n\times n$-Matrix $A=(a_{ij})$ heißt \underline{symmetrisch}, wenn
    \[
    a_{ij}=a_{ji}\qquad\forall i,j=1,\dots,n
    \]
\end{definition}
\begin{theorem}
    Sei $V$ ein reeller euklidischer endlichdimensionaler Vektorraum, $T:V\to V$ linear und $\{v_1,\dots,v_n\}$
    eine Orthonormalbasis von $V$. Dann gilt:

    $T$ ist selbstadjungiert $\biimplies$ Die darstellende Matrix $A$ von $T$ ist symmetrisch.
\end{theorem}