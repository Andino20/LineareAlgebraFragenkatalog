\section{Frage 18}
\textit{Die Determinante: Definition. Formulieren Sie die wichtigsten Eigenschaften der Determinante.}

\begin{definition}[Determinante]
    Sei $A=(a_{ij})\in M(n\times n)$. Wir definieren die Determinanten von 
    $A$ (kurz: det($A$)), folgendermaßen induktiv:\\
    \underline{n=1}: $\mathrm{det}(a) := a$.\\
    \underline{n>1}: Entwicklung nach der $j$-ten Spalte.

    Sei $j$ mit $1\leq j \leq n$ eine beliebige, aber fixe Zahl.
    \[
    \mathrm{det}(A):= \sum_{i=1}^{n}(-1)^{i+j}\cdot a_{ij} \cdot \mathrm{det}(A_{ij})\]
\end{definition}

\begin{theorem}
    Die Abbildung
    \begin{align*}
        \mathrm{det}: &\ M(n\times n)\to \mathbb{R}\\
        &\ A\mapsto \mathrm{det}(A)
    \end{align*}
    ist eine eindeutig bestimmte Abbildung, die folgende Eigenschaften erfüllt:
    \begin{enumerate}
        \item $\mathrm{det}$ ist linear in jeder Zeile, d.h. sei $a_i=(a_{i1},\dots,a_{in})$ die $i$-te Zeile
        von $A$. Dann gilt
        \[
            \mathrm{det}\begin{pmatrix}
                \vdots\\
                \lambda a_i' + \mu a_i''\\
                \vdots
            \end{pmatrix} = \lambda\det \begin{pmatrix}
                \vdots\\a_i'\\\vdots
            \end{pmatrix} + \mu\det \begin{pmatrix}
                \vdots\\a_i''\\\vdots
            \end{pmatrix}\qquad (\lambda, \mu\in\mathbb{R})
        \]
        \item Ist $\mathrm{rg}(A) < n\implies\det A  = 0$.
        \item $\det E  = 1$, wobei $E$ die Einheitsmatrix ist.
    \end{enumerate}
\end{theorem}
\begin{definition}[Elementare Zeilenumformungen]
    \[
        A \leadsto A'
    \]
    \begin{enumerate}
        \item[(I)] Vertauschen von 2 Zeilen:
        \[
            \det A' = -\det A
        \]
        \item[(II)] Multiplikation mit $\lambda\neq 0$:
        \[
            \det A'=\lambda\cdot \det A
        \]
        \item[(III)] Addition eines Vielfachen einer Zeile zu einer anderen:
        \[
            \det A' = \det A
        \]
    \end{enumerate}
\end{definition}
\begin{definition}[Determinante der transponierten Matrix]
    Sei $A=(a_{ij})\in M(n\times n)$
    \[
        A^T := (a_{ij}^T)\text{ mit } a_{ij}^T:= a_{ji}
    \]
    heißt die zu $A$ transponierte Matrix.

    Man erhält $A^T$ aus $A$, indem man die Elemente von $A$ an der Hauptdiagonale spiegelt. Dann gilt folgende
    Eigenschaft:
    \[
        \det(A) = \det (A^T)
    \]
\end{definition}
\begin{definition}
    [Multiplikationssatz] Seien $A,B\in M(n\times n)$. Dann gilt:
    \[
        \det(AB)=(\det A)(\det B)
    \]
\end{definition}
\begin{definition}
    Sei $A\in M(n\times n)$
    \[
        A \text{ ist invertierbar }\Longleftrightarrow \det A \neq 0
    \]
\end{definition}
\begin{definition}
    Sei $A$ eine obere Dreiecksmatrix, d.h.
    \[
    A=\begin{pmatrix}
        a_{11} & & *\\
         & \ddots\\
        0 & & a_{nn}
    \end{pmatrix}
    .\]
    Dann gilt
    \[
        \det A = a_{11}\cdot a_{22}\cdots a_{nn}
    \]
\end{definition}
\begin{definition}
    [Berechnungsverfahren für die Determinante] Verwandle $A$ durch elementare Umforungen in eine obere
    Dreiecksmatrix $A'$. Dies ist sicher möglich, denn sonst wäre $\mathrm{rg}\ A < n$ und damit $\det A = 0$.

    Sei $r$ die Anzahl der Zeilenumformungen
    \[
        \implies \det A = (-1)^r\det A'= (-1)^r a_{11}'\cdots a_{nn}'
    \]
\end{definition}
\begin{theorem}
    Sei $A\in M(n\times n), b\in \mathbb{R}^n$. Dann ist das $n\times n$-Gleichungssystem
    \[
        Ax = b
    \] genau dann \underline{eindeutig} lösbar, wenn
    \[
        \det A \neq 0.
    \]
\end{theorem}