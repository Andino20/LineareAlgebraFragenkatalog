\section{Frage 20}
\textit{Eine lineare Abbildung auf einem endlichdimensionalem komplexen Vektorraum hat mindestens einen Eigenwert:
Beschreiben Sie die wichtigsten Schritte im Beweis des Satzes.}
\begin{theorem}
    Sei $V$ ein endlichdimensionaler Vektorraum \underline{über $\mathbb{C}$} und $T:V\to V$ linear. Dann hat
    $T$ mindestens einen Eigenwert $\lambda\in\mathbb{C}$.
\end{theorem}
\begin{proof}
    Sei $B=\{v_1,\dots,v_n\}$ eine Basis von $V$ und $A=(a_{ij})\in M(n\times n)$ die darstellende Matrix
    von $T$ bezüglich $B$. Sei weiters 
    \begin{align*}
        I : V&\to v\\
        x &\mapsto x
    \end{align*}
    die identische Abbildung.\\
    \rule{\textwidth}{0.2px}\\
    \underline{Schritt 1}: Die Determinante des charakteristischen Polynoms herleiten.
    \begin{align*}
        \lambda\in \mathbb{C} \text{ Eigenwert } &\biimplies \exists x\neq 0 : T(x)=\lambda x\\
        &\biimplies T(x) - \lambda x = 0\\
        &\biimplies (T-\lambda I) (x) = 0\\
        &\biimplies x\in \mathrm{ker}(T-\lambda I) \land x\neq 0\\
        &\biimplies\mathrm{ker}(T-\lambda I)\neq 0\\
        &\biimplies\mathrm{dim}(\mathrm{ker}(T-\lambda I)) \geq 1 > 0\\
        \text{(Einschub Dimensionsformel): } &\begin{cases}
            \mathrm{dim}(\mathrm{ker}\ T) + \mathrm{dim}(\mathrm{im}\ T) = \mathrm{dim}\ V\\
            \mathrm{dim}(\mathrm{im}\ T)=\mathrm{rg}\ A
        \end{cases}\\
        &\biimplies\mathrm{dim}(\mathrm{im}\ (T-\lambda I)) \leq n-1 < n\\
        &\biimplies\mathrm{rg}(A-\lambda E) \leq n-1 < n\\
        &\biimplies\det (A- \lambda E) = 0.\\
        \underline{\lambda\in \mathbb{C} \text{ Eigenwert } }&\underline{\biimplies \det (A- \lambda E) = 0}
    \end{align*}
    \underline{Schritt 2}: Das charakteristische Polynom.
    
    Seien $A,B\in M(n\times n) \implies \exists c_0, c_1, \dots, c_n\in\mathbb{K}$, sodass
    \begin{gather*}        
        \det(A - \lambda B) = c_n \lambda^n + \dots + c_1 \lambda + c_0\\
        \implies \det (A-\lambda E) = (-1)^n \lambda^n + a_{n-1}\lambda^{n-1} + \dots + a_1 \lambda + a_0
    \end{gather*}
    \underline{Schritt 3}: Über den Fundamentalsatz der Algebra eine Aussage über die Nullstellen des charakteristischen Polynoms treffen.

    Jedes komplexe Polynom
    \[
        P(z) = c_nz^n + \dots + c_1z + c_0\qquad c_0,\dots,c_1\in\mathbb{C}
    \]
    von Grad $n\ (n \geq 1)$ hat mindestens eine Nullstelle, d.h. es gibt ein $z_0\in\mathbb{C}$ mit $P(z_0)=0$.\\
    \underline{Schritt 4}: Das charakteristische Polynom ist ein komplexes Polynom vom Grad $n$. Somit gibt es nach
    dem Fundamentalsatz der Algebra zumindest eine Nullstelle
    \begin{align*}
        \implies& \exists \lambda\in \mathbb{C} \text{ mit } \det(A-\lambda E) = 0\\
        \biimplies& \lambda \text{ ist Eigenwert von } T
    \end{align*}
    Also hat $T$ zumindest einen Eigenwert.
\end{proof}