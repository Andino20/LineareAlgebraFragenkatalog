\section{Frage 6}
\textit{Beispiele zu linearen Gleichungssystemen.}
\begin{example}
    Lineares Gleichungssystem als eine Menge an Gleichungen.
    \begin{gather*}
        \sysdelim..
        \systeme{
            2x_1 + x_2 - 2x_3 + 3x_4 = 4,
            3x_1 + 2x_2 - x_3 + 2x_4 = 6,
            3x_1 + 3x_2 + 3x_3- 3x_4 = 6
        }
    \end{gather*}
    \begin{elimination}[3]{4}{1.5em}{1.5}
        \eliminationstep
        {
            2 & 1 & -2 & 3 & 4\\
            3 & 2 & -1 & 2 & 6\\
            3 & 3 & 3 & -3 & 6
        }
        {
            \frac{1}{2}\cdot \text{I}
        }
        \eliminationstep
        {
            1 & \frac{1}{2} & -1 & \frac{3}{2} & 2\\
            3 & 2 & -1 & 2 & 6\\
            3 & 3 & 3 & -3 & 6
        }
        {
            (-3)\text{I} + \text{II} \to \text{II}\\
            (-3)\text{I} + \text{III} \to \text{III}
        }
        \\[10pt]
        \eliminationstep
        {
            1 & \frac{1}{2} & -1 & \frac{3}{2} & 2\\
            0 & \frac{1}{2} & 2 & -\frac{5}{2} & 0\\
            0 & \frac{3}{2} & 6 & -\frac{15}{2} & 0
        }
        {
            2\cdot\text{II}
        }
        \eliminationstep
        {
            1 & \frac{1}{2} & -1 & \frac{3}{2} & 2\\
            0 & 1 & 4 & -5 & 0\\
            0 & \frac{3}{2} & 6 & -\frac{15}{2} & 0
        }
        {
            (-\frac{3}{2})\text{II} + \text{III} \to \text{III}
        }
        \\[10pt]
        \eliminationstep
        {
            1 & \frac{1}{2} & -1 & \frac{3}{2} & 2\\
            0 & 1 & 4 & -5 & 0\\
            0 & 0 & 0 & 0 & 0
        }
        {
            (-\frac{3}{2})\text{II} + \text{I} \to \text{I}
        }
        \eliminationstep
        {
            1 & 0 & -3 & 4 & 2\\
            0 & 1 & 4 & -5 & 0\\
            0 & 0 & 0 & 0 & 0
        }
        {
            \\
        }
    \end{elimination}
    \[
        \implies x = \begin{pmatrix}
            2\\0\\0\\0
        \end{pmatrix} + \lambda_1 \begin{pmatrix}
            3\\-4\\1\\0
        \end{pmatrix} + \lambda_2 \begin{pmatrix}
            -4\\5\\0\\1
        \end{pmatrix}
    \]
\end{example}
\begin{example}
    $Ax=b$ mit
    \[
    A = \begin{pmatrix}
        4 & 1 & -1\\
        -2 & 0 & 3\\
        2 & 1 & 2
    \end{pmatrix}, b = \begin{pmatrix}
        0\\0\\1
    \end{pmatrix}
    \]
    \begin{elimination}[3]{3}{1.1em}{1.5}
        \eliminationstep
        {
            4 & 1 & -1 & 0\\
            -2 & 0 & 3 & 0\\
            2 & 1 & 2 & 1
        }
        {
            \frac{1}{4}\cdot \text{I}
        }
        \eliminationstep
        {
            1 & \frac{1}{4} & -\frac{1}{4} & 0\\
            -2 & 0 & 3 & 0\\
            2 & 1 & 2 & 1
        }
        {
            2\text{I} + \text{II} \to \text{II}\\
            (-2)\text{I} + \text{III} \to \text{III}
        }
        \\[10pt]
        \eliminationstep
        {
            1 & \frac{1}{4} & -\frac{1}{4} & 0\\
            0 & \frac{1}{2} & \frac{5}{2} & 0\\
            0 & \frac{1}{2} & \frac{5}{2} & 1
        }
        {
            2\cdot\text{II}
        }
        \eliminationstep
        {
            1 & \frac{1}{4} & -\frac{1}{4} & 0\\
            0 & 1 & 5 & 0\\
            0 & \frac{1}{2} & \frac{5}{2} & 1
        }
        {
            (-\frac{1}{2})\text{II} + \text{III} \to \text{III}
        }
        \\[10pt]
        \eliminationstep
        {
            1 & \frac{1}{4} & -\frac{1}{4} & 0\\
            0 & 1 & 5 & 0\\
            0 & 0 & 0 & 1
        }
        {
            \\
        }
    \end{elimination}
    Da die letzte Zeile keine $0$ in der letzten Spalte hat, ist das Gleichungssystem \underline{nicht lösbar}.
\end{example}